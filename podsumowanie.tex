\chapter{Podsumowanie}
  
  W poprzednim rozdziale omówiono szczegóły implementacyjne przyjętego rozwiązania oraz krótko omówiono jego wady i zalety z perspektywy autora. Ten rozdział zawiera proponowane ścieżki rozwoju stworzonego oprogramowania oraz podsumowanie całości pracy.

  \section{Plan rozwoju}
  
    Proponowane narzędzie wsparcia gromadzenia wymagań, jest jedynie próbą nadania kierunku rozwojowi oprogramowania z tego segmentu. W pełni funkcjonalne narzędzie powinno wspierać obsługę wielu kont użytkowników, śledzenie wymagań, pełną obsługę diagramów (aktorzy, połączenia przypadków użycia, dziedziczenie), wsparcie importu/eksportu danych oraz automatyczny, asynchroniczny zapis obszaru roboczego diagramu. 
    
    Na szczególną uwagę zasługuje koncepcja integracji narzędzia z systemem kontroli wersji. Programista pracujący nad kodem źródłowym dotyczącym wybranego wymagania, mógłby w komentarzu zamykającym zadanie umieszczać identyfikator zrealizowanego wymagania. W systemie Reqmanager należałoby zaimplementować podsystem monitorujący wszystkie operacje zapisu do repozytorium, ekstrahować z komentarzy identyfikatory wymagań i odpowiednio oznaczać status ich realizacji na podstawie treści komentarza, umieszczając przy wymaganiu informacje z systemu kontroli wersji (programista, komentarz, data, status, etc).

    Kolejną możliwością rozwoju jest lepsze wsparcie dokumentowania przypadków użycia poprzez wzbogacenie możliwości edycji warunków zajścia a priori i a posteriori. Przykładowo warunki takie mogłyby być zapisywane w postaci kodu źródłowego Groovy lub Java i wykonywane po stronie serwera w odpowiednim momencie. Taka funkcjonalność umożliwiałaby nowatorskie podejście do testowania systemów oraz mogłaby dawać początek kierunku rozwoju bardziej zaawansowanego systemu umożliwiającego dynamiczne generowanie kodu poprzez definicję samych wymagań.

    Interesującą koncepcją jest również procesowanie języka naturalnego i automatyczna ekstrakcja wymagań z dokumentów tekstowych. Takie rozwiązanie musiałoby opierać się o zaawansowane biblioteki analizujące tekst oraz implementację inteligentnych algorytmów radzących sobie z ekstrakcją najistotniejszych fragmentów tekstu w celu sformułowania logicznego wymagania.

    
  \section{Zakończenie}

    Analiza istniejących rozwiązań, doświadczenia wyniesione z licznych, małych projektów komercyjnych realizowanych przez sektor MŚP oraz badania naukowe w tej dziedzinie, pokazują, że inżynieria wymagań jest w większości organizacji bardzo słabo rozwinięta. Jednak poziom skomplikowania procesorow i sprzętu komputerowego będzie rozwijał się w zatrważającym tempie jeszcze przez wiele lat, a wraz z nimi, potrzeba tworzenia coraz bardziej złożonego, niezawodnego oprogramowania. Dlatego organizacje chcące przetrwać na tak konkurencyjnym rynku jakim jest wytwarzanie oprogramowania, muszą pracować nad ciągłym usprawnianiem procesów, budową odpowiedniej infrastruktury i świadomości wśród pracowników oraz pracować blisko ze środowiskiem naukowym w poszukiwaniu coraz lepszych rozwiązań inżynieryjnych. Jedno narzędzie czy koncepcja nie rozwiąże problemów z jakimi mamy do czynienia w inżynierii wymagań. Jest to dziedzina integrująca bardzo zróżnicowane środowiska - świat naukowy i techniczny, ze światem biznesowym, opartym często na bardzo ciężko mierzalnych parametrach. Z tego względu niezwykle trudno jest wyznaczyć jasne granice inżynierii wymagań i odpowiednio sformalizować procesy. Być może w przyszłości, zostaną rozwinięte zaawansowane metody formalne pozyskiwania wymagań lub powstaną systemy procesowania języka naturalnego, które dadzą początek znacznyn przemianom i zniwelują zależność od czynnika ludzkiego w procesach wytwórzczych. 

    Od formalnego wprowadzenia dziedziny inżynierii oprogramowania, jako dziedziny nauki, w roku 1968, na konferencji NATO, wypracowano wiele skutecznych metod wspierającyych procesy związane z pozyskiwaniem, przetwarzaniem i zarządzaniem wymaganiami. Dojrzałość organizacji tworzących oprogramowanie zależy między innymi od poziomu wiedzy związanej z procesami inżynierii wymagań oraz umiejętności ich skutecznego wdrożenia. Narzędzia zastosowane w tych procesach grają istotną, jednak nie pierwszoplanową rolę. Nadal na dzień dzisiejszy kluczem do sukcesu projektu, oprócz odpowiedniego przygotowania merytorycznego jest umiejętność zbudowania odpowiedniego zespołu specjalistów z doświadczeniem i wyposażenie ich w odpowiednie mechanizmy pozwalające na swobodne i efektywne działanie. Organizacje nadal starające się przełożyć bezpośrednio dziedzinę wytwarzania oprogramowania na klasyczne zasady inżynierii systemów zdają się niedostrzegać lub ignorować drastyczne różnice jakie dzielą te dziedziny. Kluczem do sukcesu projektów informatycznych, w pierwszej kolejności, są doskonali, doświadczeni projektanci.

$ \blacksquare $

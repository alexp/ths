\chapter{Wstęp}

    Rozwój technologii internetowych w ostatnich latach umożliwił programistom tworzenie wysoce interaktywnych narzędzi przy jednoczesnym ograniczeniu wymagań systemowych jakie muszą spełnić klienci, chcący skorzystać z wytworzonego oprogramowania. W szczególności, niedawno wprowadzona na rynek, technologia HTML5 daje nowe, szerokie możliwości aplikacjom internetowym. Rozwiązania, które były dotychczas zarezerwowane głównie dla technologii Flash, MS Silverlight czy JavaFX stają się wspierane natywnie, w każdej popularnej przeglądarce. 

    Z wymaganiami stykamy się praktycznie od samego początku pracy nad projektem informatycznym. Niezbędne są skuteczne narzędzia gromadzenia i przetwarzania wymagań oraz komunikowania ich wszystkim członkom zespołu. Dzięki wyżej wspomnianej technologii HTML5, istnieje możliwość usprawnienia procesów fazy analizy przy jednoczesnym zachowaniu wysokiej dostępności aplikacji, wprowadzając nowe narzędzia oparte na metodach graficznych. 

    Rynek nowych technologii roi się dzisiaj od aplikacji rozwiązujących rozmaite problemy, od prostych list zakupów, po zaawansowane narzędzia wspierające zarządzanie projektami. Pomimo istnienia wielu aplikacji skupiających się na procesach inżynierii wymagań, bardzo ciężko znaleźć rozwiązanie dostosowane do potrzeb i realiów prowadzenia projektów w małych i średnich przedsiębiorstwach. 
          
    W niniejszej pracy zostanie zaprezentowana koncepcja aplikacji wspierającej zarządzanie wymaganiami, stworzona z myślą o rzeczywistych problemach jakie wiążą się z tym etapem projektu informatycznego. Zaproponowany system ma za zadanie ułatwić zespołom projektowym komunikację i dostęp do wiedzy przy jednoczesnym skupieniu uwagi na najistotniejszych aspektach definiowania i przetwarzania wymagań.


    \section{Cel pracy}

      W małych i średnich przedsiębiorstwach, proces gromadzenia i dokumentowania wymagań użytkowników jest najczęściej słabo sformalizowany. Wymagania często są przekazywane werbalnie lub trafiają do zespołu z różnych, heterogenicznych źródeł i w wielu, zwykle niekompatybilnych, formatach. Pomimo faktu, iż na rynku nie brakuje oprogramowania wspierającego zarządzanie wymaganiami, istniejące rozwiązania, często nie są przystosowane do realiów prowadzenia projektów w małych, dynamicznych przedsiębiorstwach. Skomplikowane i trudno dostępne systemy często wymagają instalacji oprogramowania po stronie klienta a rozwiązania dostępne online, w formule SaaS (Sofware As A Service) często powielają tylko funkcjonalności potężnych (i drogich) systemów takich jak IBM DOORS czy IBM ReqiusitePro. Ponadto niezwykle trudno jest znaleźć rozwiązanie darmowe lub posiadające ogólnodostępną wersję demonstracyjną. 

      Celem niniejszej pracy jest konstrukcja prototypu systemu usprawniającego zbieranie i przetwarzanie wymagań użytkownika w postaci tekstowej oraz graficznej, dostarczając narzędzi edycji tekstu oraz graficznego modelowania przypadków użycia. W efekcie, użytkownik, po wprowadzeniu początkowych wymagań, ma możliwość automatycznego wygenerowania dokumentu SRS (Software Requirement Specification) \cite{IEEE89}.
      
      Podejście zaprezentowane w tej pracy, opiera się na założeniach, że nowoczesne oprogramowanie wspierające zarządzanie wymaganiami, powinno m.in.: stanowić centralne repozytorium wiedzy o wymaganiach; być łatwo dostępne w jak największej ilości różnych środowisk; umożliwiać łatwą kolaborację oraz być łatwe w obsłudze, prowokując do kreatywności, zamiast stawiać bariery w postaci skomplikowanych formularzy i tabel oraz niejasnych funkcjonalności. 


    \section{Zarys koncepcji}

      Proces zbierania wymagań jest trudny, ponieważ często jest procesem niesformalizowanym, rozproszonym i udokumentowanym na wiele sposobów (różne nośniki danych, niekompatybilne oprogramowanie). Ponadto wymagania zwykle pochodzą od wielu interesariuszy: od sponsora projektu, po użytkowników końcowych. Niezależnie od poziomu dojrzałości organizacji i stosowanej metodyki zarządzania projektami, źródła pochodzenia wymagań pozostają rozproszone i niekompatybilne. 

      Na potrzeby niniejszej pracy, został stworzony prototyp systemu pozwalający w łatwy sposób dokumentować gromadzone wymagania użytkownika. W powstałym systemie, głównymi narzędziami definiowania wymagań są pliki tekstowe oraz diagramy przypadków użycia. Osoby odpowiedzialne w projekcie za zarządzanie wymaganiami, otrzymują proste w obsłudze narzędzia, pozwalające na skupienie się nad sednem problemu, który poszczególne wymaganie ma na celu rozwiązać. Zaimplementowano system zarządzania projektem, w którym użytkownik ma możliwość zdefiniowania podstawowych parametrów takich jak nazwa projektu, planowany czas zakończenia oraz ogólny opis, zawierający kluczowe informacje o projekcie na etapie analizy. Korzystając z narzędzia definiowania wymagań w projekcie, użytkownik ma do dyspozycji edytor tekstowy, w którym dokumentuje zidentyfikowane wymagania użytkownika. Dzięki wykorzystaniu prostego języka znaczników (Markdown \cite{Grub04}), użytkownik skupia się na logicznej strukturze opisu wymagania. Brak narzędzi typu ,,WYSIWYG'' znanych ze standardowych edytorów tekstu jest zabiegiem celowym - ma sprawiać, że użytkownik nie jest rozpraszany potrzebą myślenia o graficznej reprezentacji tekstu opisującego problem, przy jednoczesnym zachowaniu czytelnej struktury dokumentu. System załączników i komentarzy, umożliwia iteracyjną pracę nad wymaganiami przy wykorzystaniu zewnętrznych źródeł informacji. 

      Dzięki graficznemu edytorowi przypadków użycia standardu UML, użytkownik ma możliwość definiowania kluczowych funkcjonalności systemu i łączenia ich z wybranymi wymaganiami. Tworzone przez użytkowników diagramy, są dostępne do ponownego wykorzystania w innych miejscach w systemie, dzięki czemu wspierana jest kolaboracja i korzystanie z już istniejących rozwiązań. Zarówno opisy wymagań jak i przypisane im przypadki użycia, są integralną częścią dokumentu specyfikacji wymagań użytkownika. Dlatego na każdym etapie fazy analizy istnieje możliwość automatycznego wygenerowania specyfikacji wymagań na podstawie danych wprowadzonych do systemu. Takie podejście sprzyja modelowi iteracyjnemu tworzenia specyfikacji i umożliwia prezentację prac nad dokumentem już na wczesnym etapie projektu. 

      Koncepcja systemu bazuje na silnym przekonaniu, że brak ograniczeń w postaci skomplikowanych i przytłaczających funkcjonalności pozwala skupić się na procesie twórczym w fazie definiowania wymagań, umożliwiając rozwiązywanie rzeczywistych problemów. 

    \section{Struktura pracy}

      W rozdziale 2 zostanie przedstawiony stan faktyczny związany z zastosowaniem metod pozyskiwania, gromadzenia i zarządzania wymaganiami w małych i średnich przedsiębiorstwach. Rozdział 3 przedstawia dostępne na rynku narzędzia wspierające zarządzanie wymaganiami. Rozdział 4 jest dokładnym opisem dziedziny problemu oraz koncepcji zaimplementowanego prototypu. W rozdziale 5 zaprezentowano najistotniejsze technologie, jakie wykorzystano podczas pracy nad prototypem oraz krótkie omówienie środowiska programistycznego w jakim powstał przedmiot pracy. Rozdział 6 stanowi opis zaimplementowanego prototypu. W rozdziale 7 zawarto podsumowanie wyników pracy oraz propozycje kierunków dalszego rozwoju. 

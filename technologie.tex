\chapter{Technologie wykorzystane w implementacji}
  
  W poprzednim rozdziale opisano główne założenia proponowanego systemu i jego poszczególne funkcjonalności. W tym rozdziale zostaną opisane narzędzia, technologie i biblioteki, które zastosowano podczas budowy prototypu.

  \section{Zarys technologii}
    
    Implementację systemu oparto o technologie Java/Jee i pochodne. Jako podstawę szkieletu aplikacji wykorzystano framework Grails umożliwiający szybkie prototypowanie aplikacji internetowych. W warstwie bazy danych wykorzystano bazę postgresql oraz technologię mapowania obiektowo-relacyjnego Hibernate ORM, dostarczaną wraz z frameworkiem Grails. 
    
    Interfejs użytkownika oparto o Twitter Bootstrap - pre-definiowany zestaw stylów oraz komponentów do natychmiastowego wykorzystania w aplikacjach internetowych. 

    Moduł graficznej edycji diagramów UML został stworzony wykorzystując ,,jsUML2'' - bibliotekę przygotowaną przez zespół prof. José Raúl'a Romero, na uniwersytecie w Kordobie. 

    Edytor tekstu oparty jest o bibliotekę JavaScript Ace Editor. Jest to wiodąca biblioteka, wykorzystywana m. in. w eksperymencie Cloud9 IDE - webowym, zintegrowanym środowisku programistycznym zapoczątkowanym przez zespół Mozilli, pod nazwą Mozilla Skywrite.
  
    Pierwotnie docelowym środowiskiem testowym dla proponowanego systemu był Google App Engine, jednak ze względu na ograniczenia tej usługi, prototyp umieszczono na darmowej infrastrukturze firmy Heroku (heroku.com) znacznie lepiej obsługującej nowoczesne aplikacje oparte o Javę.

  \section{Framework Grails}
    
    Framework Grails jest stosunkowo młodą technologią opartą na sprawdzonych rozwiązaniach. Firma która zajmuje się rozwojem tej technologii jest tą samą organizacją, odpowiedzialną za stworzenie Spring Framework - jednego z najpopularniejszych i jednocześnie najbardziej rozbudowanych platform programistycznych dla języka Java dostępnych na rynku. To właśnie Spring Framework leży u podstaw Grails i wraz z Hibernate ORM stanowi trzon technologiczny tego narzędzia.

   Grails integruje najlepsze praktyki i narzędzia zarówno ze Spring Framework jak i z Hibernate ORM. Jest poniekąd odpowiedzią środowiska Javy, na rosnącą konkurencyjność frameworków opartych o dynamicznie typowane, skryptowe języki programowania jak Django (język python) czy Ruby On Rails (Ruby). Jednocześnie, ze względu na fakt iż jest swoistą nakładką na technologie oparte na Javie, dostarcza znacznie większych możliwości niż konkurenci, wynikających z rozbudowanego ekosystemu Javy.


    \subsubsection{Język Groovy}

      Podstawowym językiem programowania w platformie Grails, jest język Groovy. Groovy jest dynamicznie kompilowanym językiem o składni i filozofii zbliżonej do takich języków jak Python lub Ruby. Kod bajtowy będący wynikiem kompilacji uruchamiany jest na wirtualnej maszynie Javy, przez co język integruje się niemal w sposób przeźroczysty z technologiami opartymi o JVM. Ponadto, kod programu napisanego w Javie jest poprawnym programem Groovy zarówno pod względem syntaktycznym jak i semantycznym. Bardziej zwięzła i przyjazna składnia Groovy wraz z szerokimi możliwościami Javy sprawia, że ten język skryptowy jest solidnym rozwiązaniem o szerokim spektrum zastosowań.

      W celach porównawczych, poniżej zamieszczono znany, trywialny problem programistyczny ,,FizzBuzz'', zarówno w języku Groovy, jak i Java. Program ,,FizzBuzz'' otrzymując na wejściu ciąg liczb, wypisuje na konsolę słowo ,,Fizz'' jeśli dana liczba jest podzielna przez 3, słowo ,,Buzz'', jeśli dana liczba jest podzielna przez 5, natomiast słowo ,,FizzBuzz'' w przypadku podzielności przez 15.

      \begin{lstlisting}[caption={program FizzBuzz}, label={lst:FizzBuzz}]

      // Groovy:
      for (i in 1..100) {
        println "${i%3?'':'Fizz'}${i%5?'':'Buzz'}" ?: i
      }    

      //Java:
      public class FizzBuzz{
        public static void main(String[] args){
          for(int i= 1; i <= 100; i++){
            if(i % 15 == 0){
              System.out.println("FizzBuzz");
            } else if(i % 3 == 0){
              System.out.println("Fizz");
            } else if(i % 5 == 0){
              System.out.println("Buzz");
            } else{
              System.out.println(i);
            }
          }
        }
      }

    \end{lstlisting}

    \subsubsection{Grails}

    W szczególności filozofia Grails jest bardzo zbliżona do popularnego frameworka Ruby On Rails (RoR) umożliwiającego ekspresowe prototypowanie aplikacji obsługujących komunikację z bazą danych w zakresie (1) tworzenia, (2) odczytu, (3) aktualizacji i (4) usuwania danych za pomocą formularzy. Tego typu aplikacje popularnie nazywane są aplikacjami ,,CRUD'' - create, retrieve, update, delete). Wiele przyjętych rozwiązań w Grails zostało zaczerpniętych z RoR, jednak oba frameworki znacznie różnią się pod względem technologicznym. Szczegółowe porównanie tych technologii leży poza zakresem tej pracy, jednak osoby zainteresowane, powinny zapoznać się z doskonałym wątkiem w serwisie Stackoverflow, przytoczonym w bibliografii \cite{RailsGr}.

    Grails jest silnie oparte o paradygmat ,,convention over configuration'' - architekturze zakładającej minimalizację potrzeby konfiguracji aplikacji przez programistę na rzecz przyjętych konwencji, do których należy się stosować, aby wydajnie dostarczać działające rozwiązania. Podejście to dotyczy wszelkich aspektów związanych z tworzeniem aplikacji w Grails - od struktury katalogów, przez umiejscowienie i zawartość plików konfiguracyjnych, po nazewnictwo klas, zmiennych i zastosowanie odpowiednich struktur danych. 

    \subsubsection{Model - View - Controller}

    Oparcie warstwy webowej w Grails na wzorcu Model-View-Controller (MVC) jest naturalnym podejściem w nowoczesnych frameworkach ułatwiających tworzenie aplikacji internetowych. Wzorzec MVC \cite{GoF} pierwotnie dotyczył tworzenia aplikacji okienkowych w Smalltalk'u-80 \cite{coad93}. Podstawowym założeniem MVC jest separacja danych i logiki od ich reprezentacji graficznej. MVC składa się z trzech rodzajów obiektów. 
  
    Model jest reprezentacją danych, wiedzy. Model może być odpowiednikiem jakiegoś obiektu lub strukturą wielu obiektów. 

    Widok jest wizualną reprezentacją modelu, ,,pobiera'' dane na temat obiektu i wyświetla je w odpowiedni sposób, często uwydatniając pewne elementy modelu, inne z kolei odpowiednio ukrywając. Funkcja ta czyni z widoku pewnego rodzaju filtr prezentacji danych. 

    Kontroler stanowi łącznik pomiędzy użytkownikiem a systemem, z którym użytkownik wchodzi w interakcję. Dostarcza użytkownikowi odpowiednich widoków, adekwatnych do stanu systemu \cite{Trygve79}.

    \begin{figure*}[t]
      \centering
      \includegraphics[width=1.0\textwidth]{img/mvc-1.pdf}
      \caption{Model-View-Controller - różna reprezentacja graficzna tych samych danych. Żródło: \cite{GoF}.}
      \label{fig:mvc-1}
    \end{figure*}

    \subsubsection{Praca z Grails}

    Struktura katalogów nowo utworzonej aplikacji Grails wygląda następująco:

    \begin{verbatim}

    + grails-app
      + conf           ---> lokalizacja artefaktów konfiguracji
        + hibernate    ---> opcjonalne pliki konfiguracyjne hibernate
        + spring       ---> opcjonalne pliki konfiguracyjne spring 
      + controllers    ---> kontrolery aplikacji
      + domain         ---> klasy reprezentujące modele
      + i18n           ---> zlokalizowane komunikaty il8n
      + services       ---> warstwa serwisowa
      + taglib         ---> biblioteki znaczników
      + util           ---> klasy pomocnicze special utility classes 
      + views          ---> lokalizacja widoków
        + layouts      ---> lokalizacja 
      + lib
      + scripts        ---> skrypty gant
      + src
      + groovy         ---> opcjonalnie, źródła groovy inne niż
                            w grails-app/*
      + java           ---> opcjonalnie, źródła java 
      + test           ---> generowane testy aplikacji
      + web-app
        + WEB-INF

    \end{verbatim}

    \subsubsection{GORM - Grails Object Relational Mapping}
  
    Po wstępnej analizie, mając wiedzę na temat dziedziny problemu, w prosty sposób można utworzyć podstawową funkcjonalność przykładowej aplikacji. Proces tworzenia elementów budowanego systemu rozpoczyna się od stworzenia klasy Groovy reprezentującej model (na przykładzie implementacji z systemu Reqmanager):

    \begin{lstlisting}[caption={klasa modelu Project.groovy}, label={lst:modelExample}]
    package pl.edu.pjwstk.reqmanager.examples
    class Project {
        String name
        String description
        String mainNote
        java.sql.Timestamp timestamp
        java.util.Date deadline

        static hasMany = [requirements:Requirement]
    }
    \end{lstlisting}

    Jeśli nie skonfigurujemy aplikacji w niestandardowy sposób, po stworzeniu powyższej klasy Groovy w katalogu grails-app/domain, Framework Grails przy pomocy Hibernate utworzy następującą strukturę w bazie danych:

    \begin{lstlisting}[caption={struktura bazy danych dla modelu Project}, label={lst:dbSchemaProject}]
                      Table "public.project"
       Column    |            Type             | Modifiers 
    -------------+-----------------------------+-----------
     id          | bigint                      | not null
     version     | bigint                      | not null
     deadline    | timestamp without time zone | 
     description | character varying(255)      | not null
     name        | character varying(255)      | not null
     timestamp   | timestamp without time zone | not null
     main_note   | character varying(255)      | 
    Indexes:
        "project_pkey" PRIMARY KEY, btree (id)
    Referenced by:
        TABLE "requirement" CONSTRAINT "fk15a8dc43ffd118f2" 
        FOREIGN KEY (project_id) REFERENCES project(id)
    \end{lstlisting}

    Powyższy przykład przedstawia, jak w prosty sposób, przy niskich nakładach pracy, programista jest w stanie zbudować szkielet aplikacji zasilanej bazą danych. Jak widać, pola klasy Groovy, zostały odpowiednio odwzorowane w strukturze danych wraz z odpowiednimi typami danych po stronie bazy.

    Na szczególną uwagę zasługuje konstrukcja w klasie Project: 

    \begin{lstlisting}[caption={relacja jeden-do-wielu}, label={lst:oneToMany}]
      static hasmany = [requirements:requirement]
    \end{lstlisting}

    W języku Groovy konstrukcja ,,[:]'' jest reprezentacją pustej tablicy asocjacyjnej - struktury danych umożliwiającej przechowywanie kolekcji par ,,klucz - wartość''. W Groovy, domyślną kolekcją inicjalizowaną przy pomocy literału ,,[:]'' jest obiekt klasy java.util.LinkedHashMap. Zatem powyższa zmienna statyczna jest referencją do obiektu klasy LinkedHashMap, zawierającego jedną parę, gdzie kluczem jest ciąg znaków ,,requirements'' a wartością, jest klasa Requirement. W ten sposób Grails obsługuje relacje jeden-do-wielu. W analogiczny sposób, istnieje możliwość definiowania pozostałych rodzajów relacji, tj. jeden-do-jeden, wiele-do-wielu, etc). Konstrukcja taka, umożliwia uzyskanie listy wymagań w projekcie: 

    \begin{lstlisting}[caption={wymagania przypisane do projektu}, label={lst:requirements}]
      //pobiera z bazy projekt o id = 1
      def project = Project.get(1)

      //zwraca liste wymagan podlaczonych do projektu
      def reqs = project.requirements 
    \end{lstlisting}
    
      
    \subsubsection{Kontrolery i widoki}

    W celu połączenia konkretnych widoków z akcjami kontrolerów, konwencja Grails wymaga odpowiedniego nazewnictwa metod w klasach kontrolerów. Generalizując, należy stosować się do zasady, aby nazwa metody odpowiedzialnej za realizację wybranej akcji kontrolera, posiadała plik o tej samej nazwie, z rozszerzeniem .gsp, w odpowiednim katalogu w ścieżce grails-app/views. Poniżej zamieszczono przykład realizacji akcji ,,index'' w kontrolerze ,,ProjectController''. Wbudowany w Grails mechanizm odwzorowywania adresów URL i przekierowywania do odpowiednich akcji kontrolerów automatycznie wykrywa którą metodę należy wykonać. Zatem, po wskazaniu przeglądarce internetowej adresu http://reqmanager.heroku.com/project/index system, przez konwencję, wykryje konieczność wywołania metody ,,index'' w kontrolerze ,,ProjectController''. Z kolei po wykonaniu tej metody, (o ile programista nie zdecyduje inaczej) automatycznie wyświetlany jest widok index.gsp umieszczony w katalogu grails-app/views/project. W widoku, do wykorzystania dostępne są zmienne ustawione w tabeli asocjacyjnej, zwracanej przez kontroler (w tym przypadku, zmienna ,,project'' zawierająca referencję do listy wszystkich projektów w systemie). 


    \begin{lstlisting}[caption={widok i kontroler}, label={lst:contView}]
      // klasa kontrolera:
      package pl.edu.pjwstk.reqmanager

      class ProjectController {
        def index = { 
          return [projects : Project.list()]
        }   
      }

      //widok w pliku o nazwie index.gsp 
      //umieszczony w katalogu grails-app/views/project/
      <!DOCTYPE html>
      <html>
        <div>
        <h1>Project list</h1>
        <g:each in='${projects}' var='project'>
          Nazwa: ${project.name}<br />
          Deadline: ${project.deadline.format("dd-MM yy")}<br />
          Opis: ${project.description} 
        </g:each>
        </div>
      </html>

    \end{lstlisting}

    Należy jeszcze zwrócić uwagę na pewien detal w strukturze kodu źródłowego kontrolera. Grails dopuszcza dwa podejścia do deklarowania akcji - (1) za pomocą klasycznych metod oraz (2) przy użyciu domknięć (closures). Formalnie domknięcia są obiektami wiążącymi funkcję oraz środowisko w jakim ta funkcja ma działać. Jest to specyficzna konstrukcja, będącą zdefiniowanym blokiem kodu, który można przekazać jako parametr innej metodzie lub funkcji. Poniższy przykład wyjaśnia zasadę na jakiej działają domknięcia w Groovy: 
    
    \begin{lstlisting}[caption={domknięcie w Groovy}, label={lst:closure}]
      def closeAllProjects() {
        def projects = Project.list()
        projects.collect { it.open = false }
      }
    \end{lstlisting} 
    
    Powyższy kod pobiera do zmiennej ,,projects'' listę wszystkich projektów w systemie. Następnie, na liście projektów wywoływana jest metoda ,,collect''. Metoda collect (zdefiniowana w interfejsie Collection) przyjmuje jako parametr blok kodu, domknięcie. Metoda collect iteruje po wszystkich elementach zbioru na rzecz którego została wykonana i wykonuje blok kodu przekazany jej jako parametr na kolejnych elementach kolekcji, następnie zwracają listę zmodyfikowanych przez domknięcie elementów.

    \subsection{Spring Framework i Hibernate ORM}

      Sercem Grails jest Spring Framework oraz Hibernate ORM. Obie technologie są wiodące na rynku i szeroko stosowane w komercyjnych projektach dowolnych rozmiarów. Spring jest de facto zestawem narzędzi, wzorców projektowych i bibliotek realizujących ogromną ilość funkcjonalności, mających za zadanie przyśpieszyć proces wytwarzania oprogramowania oraz w pewnym zakresie zapewnić wysoką jakość tworzonych rozwiązań. Przede wszystkim jednak, jest tzw. kontenerem Inversion Of Control (odwrócenie sterowania) \cite{MFow01}. Prócz IoC Spring Framework to potężna platforma integrująca wiele rozwiązań, takich jak webowy framework Spring MVC, biblioteki i wzorce dotyczące bezpieczeństwa (Spring Security), wrapper jdbc, transakcje (Spring JDBC Templates), etc.

      Hibernate ORM jest technologią pozwalającą na mapowanie rekordów z relacyjnych baz danych, na obiekty w systemie przy pomocy plików konfiguracyjnych xml. W Grails, potrzeba konfiguracji warstwy ORM ogranicza się do minimum, a większość standardowych zadań można zrealizować konfigurując odpowiednio blok \emph{mapping} w klasie modelu.

  \section{Pozostałe technologie}
    \subsection{Postgresql}
      Trwałość danych w aplikacjach internetowych standardowo realizowana jest przez bazę danych. Na potrzeby prezentowanego prototypu, wykorzystano bazę danych Postgresql. Jest to obiektowo-relacyjna baza danych udostępniona na licencji Wolnego i Otwartego Oprogramowania. 

      Baza danych PostgreSQL implementuje znaczną część standardu SQL:2008 [!ref - http://www.postgresql.org/docs/9.1/static/features.html]

    \subsection{Javascript, jQuery i biblioteka jsUML2}

      Wiele funkcjonalności zrealizowano po stronie klienta, wykorzystując język JavaScript. W niewielkim zakresie wykorzystano popularną bibliotekę jQuery, wspierającą realizację najczęściej pojawiających się problemów w trakcie prac nad aplikacjami internetowymi. Jednak wykorzystanie jQuery w prototypie systemu jest marginalne i ogranicza się jedynie do zastosowania metody \emph{ajax()} ułatwiającej wykonywanie asynchronicznych wywołań serwera.

      \subsubsection{jsUML2}
        jsUML2 jest biblioteką stworzoną przez środowisko naukowe, na uniwersytecie w Kordobie. Jest to zaimplementowany w całości w języku JavaScript, rozbudowany zestaw klas umożliwiający tworzenie wielu rodzajów diagramów UML w przeglądarce internetowej. jsUML wykorzystuje najnowszą specyfikację języka znaczników HTML5. Obiekt \emph{canvas} oferuje szerokie spektrum możliwości programistom aplikacji internetowych. Element \emph{canvas} umożliwia programistyczne tworzenie kształtów i obrazów bitmapowych w przeglądarce internetowej. Canvas udostępnia programistom interfejs HTMLCanvasElement umożliwiający modyfikację zawartości elementów canvas w htmlu. Metoda interfejsu \emph{getContext()} zwraca tzw. ,,drawing context'' - w zależności od parametru przekazanego metodzie \emph{getContext()}, zwracany jest obiekt klasy (1) CanvasRenderingContext2D w przypadku przekazania paramteru '2d' lub (2) WebGLRenderingContext w przypadku przekazania paramteru 'experimental-webgl'.

        Poniżej zamieszczono przykładowy kod, umożliwiający rozpoczęcie pracy obiektem klasy CanvasRenderingContext2D. Wykonanie kodu spowoduje narysowanie prostokąta:

      \begin{lstlisting}[caption={przykład HTML5 canvas}, label={lst:canvas1}]
      var canvas = document.getElementById('example');
      var drawingContext = canvas.getContext('2d')

      drawingContext.fillStyle = "rgb(200,0,0)";  
      drawingContext.fillRect(10, 10, 55, 50);
      \end{lstlisting}

        Biblioteka jsUML2 składa się z dwóch pakietów klas: (1) \emph{UDCore} - zawierający bazowe klasy umożliwiające tworzenie i obsługę graficzną diagramów i komponentów pomocniczych; (2) \emph{UDModules} - zawierający faktyczną implementację diagramów standard UML, zbudowaną na bazie klas pierwotnych z pakietu \emph{core}.

        Głównymi elementami w poszczególnych pakietach są:

        \begin{itemize}
          \item UDCore
            \begin{itemize}
              \item Diagram
              \item Node (węzeł)
              \item Relation (relacja)
            \end{itemize}
          \item Modules 
            \begin{itemize}
              \item Use case diagrams
              \item Class diagrams
              \item Component diagrams
              \item Message sequence diagrams
              \item State charts
            \end{itemize}
        \end{itemize}

        Na listingu \ref{lst:UDCore} zawarto wykaz klas dostępnych w pakiecie \emph{core}:

        \begin{lstlisting}[caption={klasy bazowe jsUML2}, label={lst:UDCore}]
        AttributeFields.js
        AttributeItem.js
        CircleSymbol.js
        CollapsibleFields.js
        Component.js
        ComponentSymbol.js
        ConnectorItem.js
        DataStoreItem.js
        Diagram.js
        Dialog.js
        Element.js
        Elliptical.js
        GuardItem.js
        JSFun.js
        JSGraphic.js
        LoopItem.js
        NodeFigure.js
        Node.js
        ObjectItem.js
        OperationFields.js
        OperationItem.js
        Point.js
        Rectangular.js
        RegionItem.js
        Region.js
        RegionLine.js
        RelationEnd.js
        Relation.js
        RelationLine.js
        Rhombus.js
        RoleItem.js
        Rombo.js
        Separator.js
        Space.js
        SpecificationItem.js
        StereotypeFields.js
        StereotypeItem.js
        SuperComponent.js
        SuperNode.js
        Tab.js
        TextArea.js
        TextBox.js
        TextFields.js
        Text.js
        TransitionItem.js
        \end{lstlisting}

        W pakiecie modules, do dyspozycji programisty znajdują się następujące podpakiety klas:

        \begin{lstlisting}[caption={pakiety modułów jsUML2}, label={lst:UDModules}]
        activity
        class
        component
        generic
        profile
        sequence
        stateMachine
        usecase
        \end{lstlisting}

        W celu inicjalizacji biblioteki jsUML2, należy stworzyć odpowiednią strukturę dokumentu html:

        \begin{lstlisting}[caption={HTML canvas element}, label={lst:HTML5Ex}]
        <div id="ud_diagram_div" style="position: relative">
          <canvas id="c1" class="ud_diagram_canvas" style="position: absolute;" width="500")
          <canvas id="c2" class="ud_diagram_canvas" style="position: absolute;" width="500")
        </div>
        \end{lstlisting}
      
        Następnie, niezbędna jest inicjalizacja odpowiedniego diagramu (na przykładzie diagramu przypadków użycia):

        \begin{lstlisting}[caption={HTML canvas element - JS}, label={lst:jsCanvasEx}]
        var mainEl = document.getElementById('c1');
        var motionEl = document.getElementById('c2');

        var mainCtx = mainEl.getContext('2d');
        var motionCtx = motionEl.getContext('2d');

        var div = document.getElementById('ud_diagram_div')

        var diagram = new UMLUseCaseDiagram()

        //metoda initialize() przyjmuje kolejno parametry:
        //(1) unikalny identyfikator diagramu
        //(2) element blokowy html bedacy kontenerem diagramu
        //(3) obiekt 'context' stanowiacy 'gorna' warstwe
        //    diagramu
        //(4,5) rozmiary w pikslach 

        diagram.initialize(1, div, motionCtx, 600, 600) 
        \end{lstlisting}

\subsection{System kontroli wersji git i serwer heroku}
  
  W trakcie pracy wykorzystano system kontroli wersji \emph{git} oraz serwer heroku. 

  Git jest systemem kontroli wersji udostępnionym na licencji GPL, stworzonym przez Linusa Torvaldsa w 2005 roku. Głównymi założeniami projektu były szybkość działania, bezpieczeństwo oraz rozwiązanie problemów jakie twórca uważał są najbardziej rażące w systemach CSV oraz SVN.

  Heroku udostępnia platformę hostingową obsługującą wiele języków programowania. Zasada działa serwerów Heroku, oparta jest na koncepcji utrzymania aplikacji w ,,chmurze'' serwerów - infrastruktura sprzętowa jest niewidoczna dla użytkowników. Użytkownicy tej usługi wykupują moc obliczeniową serwerów odpowiadającą potrzebom utrzymywanych aplikacji, płacąc za czas dostępu do serwerów o wybranej mocy obliczeniowej. Główną zaletą tego rozwiązania jest możliwość natychmiastowego skalowania aplikacji w zależności od zapotrzebowania na moc obliczeniową. W przypadku znacznego wzrostu ruchu na serwerze, użytkownik może wykupić dodatkową moc obliczeniową, bez konieczności przerywania ciągłości działania usługi w sieci. Heroku udostępnia swoją infrastrukturę w podstawowym zakresie, bez konieczności uiszczania opłat, dlatego jest to rewelacyjne rozwiązania na potrzeby testów i prezentacji aplikacji internetowych. Dzięki natywnemu wsparciu systemu kontroli wersji Git, instalacja aplikacji w chmurze, polega na przesłaniu na serwer aktualizacji kodu źródłowego i wykonania polecenia inicjalizującego instalację kodu na serwerze heroku:


        \begin{lstlisting}[caption={}, label={lst:jsCanvasEx}]
        ~$ git add .
        ~$ git commit -m 'update'
        ~$ git push origin master
        \end{lstlisting}

\subsection{Środowisko programistyczne (Linux, vim)}

  Do stworzenia zarówno prorotypu aplikacji, jak i do edycji i składu części tekstowej niniejszej pracy, wykorzystano zasadniczo narzędzia dostępne w systemie operacyjnym Linux (Ubuntu Linux v11.10). 

  Do edycji kodu źródłowego wykorzystano klasyczny edytor tekstowy vim z zainstalowanym dodatkiem ułatwiającym poruszanie się w strukturze katalogów grails (vim grails plugin z zainstalowanym dodatkiem ułatwiającym poruszanie się w strukturze katalogów grails (vim grails plugin).

  Część opisowa niniejszej pracy również powstała w edytorze tekstu vim. Do składu tekstu zastosowano popularne oprogramowanie \LaTeX. 

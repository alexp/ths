\chapter{Stan faktyczny w inżynierii oprogramowania}

  W poprzednim rozdziale nakreślono temat przewodni niniejszej pracy. Poru\-szono problem dostępności i użyteczności istniejących narzędzi wspomagających zarządzanie wymaganiami. Zaprezentowano także autorską koncepcję rozwiązania, mającego na celu usprawnienie procesu tworzenia specyfikacji wymagań użytkownika. 

  Ten rozdział zadedykowany jest analizie stanu faktycznego w zakresie inżynierii oprogramowania w różnych organizacjach.  

  \section{Źródła wiedzy}

  Inżynieria wymagań stanowi fundament wiedzy pozwalającej na realizację wysokiej jakości projektów informatycznych w założonym budżecie i czasie. 

  Istnieje obszerna literatura podejmująca tematy stricte związane z inżynierią wymagań (Gause and Weinberg 1989; Jackson 1995; Kovitz 1999; Robertson and Robertson 1999a; Sommerville and Sawyer 1997), jednak praktyczne doświadczenia oraz liczne dostępne badania wskazują na istotne problemy z wprowadzaniem teorii w życie \cite{Boehm06, Niku00, Standish94}.

  Znany raport opublikowany przez Standish Group International \cite{Standish94} prezentuje dramatyczną rzeczywistość, gdzie tylko niewielka część projektów kończy się częściowym lub całkowitym sukcesem. Wyniki raportu wskazują na najważniejsze powody niepowodzeń projektów jak brak włączenia użytkowników w proces budowy oprogramowania, niekompletne i nieudokumentowane wymagania, wiecznie zmieniający się zakres projektów czy brak wsparcia kadry zarządzającej.      
  
  W badaniu Nikula et al. (2000) przeprowadzonym wśród małych i średnich przedsiębiorstw w Finlandii, autorzy wskazują na alarmujący brak wiedzy dotyczącej wdrażania procesów inżynierii wymagań. Jednocześnie, wnioski z badania wskazują na konieczność zajęcia się problemami inżynierii wymagań na poziomie korporacyjnym, poprzez wdrażanie, optymalizację i automatyzację procesów IW w małych i średnich przedsiębiorstwach \cite{Niku00}.
  
  Z kolei badanie Quispe et al. (2010), również dotyczące małych i średnich przedsiębiorstw, zwraca szczególną uwagę na brak lub nienależytą komunikację z klientem, adresowanie nieodpowiednich problemów klienta, wadliwe specyfikacje oraz stosowanie praktyk inżynierii wymagań ,,ad hoc'' \cite{Quispe10}.
  
  Powyższe badania zajmują się głównie czynnikami niepowodzeń projektów. W badaniu Fern\'{a}ndez et al. (2012) przeanalizowano 12 projektów zakończonych sukcesem. Wnioski z tego badania sugerują, że wiele ze zidentyfikowanych czynników wpływających na niepowodzenia projektów można z powodzeniem rozwiązać zanim staną się realnym zagrożeniem. Ponadto zauważono, że klienci często mają znaczny wpływ na parametry związane z niepowodzeniem projektu. 

  Analiza wyników powyższych badań w kontekście tej pracy sugeruje przede wszystkim konieczność zwrócenia szczególnej uwagi na poziom dojrzałości organizacji w aspekcie inżynierii wymagań. Nie wszystkie projekty kończą się porażką, a doświadczenia pokazują, że im bardziej dojrzałe procesy inżynierii wymagań w organizacji i odpowiednia ,,kultura'' firmy, tym większe są szanse na realizację projektu zwieńczoną sukcesem. 

  Podsumowując, żaden system informatyczny ani narzędzie nie rozwiąże wszystkich problemów związanych z inżynierią oprogramowania. Na organizacji wytwarzającej oprogramowanie ciąży odpowiedzialność usprawniania procesów inżynierii wymagań, edukacji pracowników i zasięgania wiedzy ekspertów. Sukcesy w branży oprogramowania można osiągnąć tylko poprzez ciągłą naukę, konsekwentną realizację procesów i ich usprawnianie na poziomie korporacyjnym.

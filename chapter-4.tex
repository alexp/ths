\chapter{Technologie wykorzystane w implementacji}
  
  krótki wstęp

  \section{Z czego korzystałem z lotu ptaka - grails, java, tomcat, heroku, git, postgresql, javascript, biblioteka jsuml2, jquery}

  \section{Framework Grails}
    
    Framework Grails jest stosunkowo młodą technologią opartą na sprawdzonych rozwiązaniach. Firma która zajmuje się rozwojem Grails, to SpringSource - ta sama organizacja, któ¶a odpowiedzialna jest za stworzenie Spring Framework - jednego z najpopularniejszych i jednocześnie najbardziej rozbudowanych platform programistycznych dla języka Java dostępnych na rynku. To właśnie Spring Framework leży u podstaw Grails i stanowi trzon technologiczny dla tego młodego narzędzia wspomagającego tworzenie aplikacji opartych o technologie webowe.  

    \subsection{Język Groovy}

      Podstawowym językiem programowania w platformie Grails, jest język Groovy. Groovy jest dynamicznie kompilowanym językiem o składni i filozofii zbliżonej do takich języków jak Python lub Ruby. Kod bajtowy będący wynikiem kompilacji uruchamiany jest na wirtualnej maszynie Javy, przez co język integruje się niemal w sposób przeźroczysty z technologiami opartymi o JVM. Ponadto, kod programu napisanego w Javie jest poprawnym programem Groovy zarówno pod względem sytnaktycznym jak i semantycznym. Bardziej zwięzła i przyjazna składnia Groovy wraz z szerokimi możliwościami Javy sprawia, że ten język skryptowy jest solidnym rozwiązaniem o szerokim spektrum zastosowań.

    \subsection{Spring Framework i Hibernate ORM}

      Sercem Grails jest Spring Framework oraz Hibernate ORM. Obie technologie są wiodące na rynku i szeroko stosowane w komercyjnych projektach dowolnych rozmiarów. Spring jest de facto zestawem narzędzi, wzorców projektowych i bibliotek realizujących ogromną ilość funkcjonalności, mających za zadanie przyśpieszyć proces wytwarzania oprogramowania oraz w pewnym zakresie zapewnić wysoką jakość tworzonych rozwiązań. Przede wszystkim jednak, jest tzw. kontenerem Inversion Of Control (odwrócenie sterowania) \cite{MFow01}. ********** tu trochę o IoC ***********. Prócz IoC Spring Framework to potężna platforma integrująca wiele rozwiązań, takich jak webowy framework Spring MVC, biblioteki i wzorce dotyczące bezpieczeństwa (Spring Security), wrapper jdbc, transakcje (Spring JDBC Templates), etc.

      Hibernate ORM jest technologią pozwalającą na mapowanie rekordów z relacyjnych baz danych, na obiekty w systemie. Jest to metoda w pewnym stopniu rozwiązująca problem niezgodności impedancji \cite{KSub01}

    \subsection{Framework Grails}

      Grails integruje wszystkie najlepsze praktyki i narzędzia zarówno ze Spring Framework jak i Hibernate ORM. Jest poniekąd odpowiedzią środowiska Javy, na rosnącą konkurencyjność frameworków opartych o dynamicznie typowane, skryptowe języki programowania jak Django (python) czy Ruby On Rails. Jednocześnie, ze względu na fakt iż jest swoistą nakładką na technologie oparte na javie, dostarcza znacznie większych możliwości niż konkurenci, wynikających z ekosystemu javy.

  \section{Pozostałe technologie}
    \subsection{Postgresql}
    \subsection{Javascript, jQuery i biblioteka jsUml2}
    \subsection{System kontroli wersji git i serwer heroku}
    \subsection{Srodowisko programistyczne (linux, vim, tmux)}

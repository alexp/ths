\chapter{Technologie wykorzystane w implementacji}
  
  W poprzednim rozdziale opisano główne założenia proponowanego systemu i jego poszczególne funkcjonalności. W tym rozdziale zostaną opisane narzędzia, technologie i biblioteki, które zastosowano podczas budowy prototypu.

  \section{Zarys technologii}
    
    Implementację systemu oparto o technologie Java/Jee i pochodne. Jako podstawę szkieletu aplikacji wykorzystano framework Grails umożliwiający szybkie prototypowanie aplikacji internetowych. W warstwie bazy danych wykorzystano bazę postgresql oraz technologię mapowania obiektowo-relacyjnego Hibernate ORM, dostarczaną wraz z frameworkiem Grails. 
    
    Interfejs użytkownika oparto o Twitter Bootstrap - predefiniowany zestaw styli oraz komponentów do natychmiastowego wykorzystania w aplikacjach internetowych. 

    Moduł graficznej edycji diagramów UML został stworzony wykorzystując ,,jsUML2'' - bibliotekę przygotowaną przez zespół prof. José Raúl'a Romero, na uniwersytecie w Kordobie [!ref]. 

    Edytor tekstu oparty jest o bibliotekę javascript Ace Editor. Jest to wiodąca biblioteka, wykorzystywana m. in. w eksperymencie Cloud9 IDE [!ref] - webowym, zintegrowanym środowisku programistycznym zapoczątkowanym przez zespoł Mozilli, pod nazwą Mozilla Skywriter [!ref].
  
    Pierwotnie docelowym środowiskiem testowym dla proponowanego systemu był Google App Engine, jednak ze względu na ograniczenia tej usługi, prototyp umieszczono na darmowej infrastrukturze firmy Heroku (heroku.com) znacznie lepiej obsługującej nowoczesne aplikacje oparte o Javę.

  \section{Framework Grails}
    
    Framework Grails jest stosunkowo młodą technologią opartą na sprawdzonych rozwiązaniach. Firma która zajmuje się rozwojem tej technologii jest tą samą organizacją, odpowiedzialną za stworzenie Spring Framework - jednego z najpopularniejszych i jednocześnie najbardziej rozbudowanych platform programistycznych dla języka Java dostępnych na rynku. To właśnie Spring Framework leży u podstaw Grails i wraz z Hibernate ORM stanowi trzon technologiczny tego narzędzia.

   Grails integruje najlepsze praktyki i narzędzia zarówno ze Spring Framework jak i z Hibernate ORM. Jest poniekąd odpowiedzią środowiska Javy, na rosnącą konkurencyjność frameworków opartych o dynamicznie typowane, skryptowe języki programowania jak Django (język python) czy Ruby On Rails (Ruby). Jednocześnie, ze względu na fakt iż jest swoistą nakładką na technologie oparte na javie, dostarcza znacznie większych możliwości niż konkurenci, wynikających z rozbudowanego ekosystemu javy.


    \subsubsection{Język Groovy}

      Podstawowym językiem programowania w platformie Grails, jest język Groovy. Groovy jest dynamicznie kompilowanym językiem o składni i filozofii zbliżonej do takich języków jak Python lub Ruby. Kod bajtowy będący wynikiem kompilacji uruchamiany jest na wirtualnej maszynie Javy, przez co język integruje się niemal w sposób przeźroczysty z technologiami opartymi o JVM. Ponadto, kod programu napisanego w Javie jest poprawnym programem Groovy zarówno pod względem sytnaktycznym jak i semantycznym. Bardziej zwięzła i przyjazna składnia Groovy wraz z szerokimi możliwościami Javy sprawia, że ten język skryptowy jest solidnym rozwiązaniem o szerokim spektrum zastosowań.

      W celach porównawczych, poniżej zamieszczono legendarny, trywialny problem programistyczny ,,FizzBuzz'', zarówno w języku Groovy, jak i Java. Program ,,FizzBuzz'' otrzymując na wejściu ciąg liczb, wypisuje na konsolę słowo ,,Fizz'' jeśli dana liczba jest podzielna przez 5, słowo ,,Buzz'', jeśli dana liczba jest podzielna przez 10, natomiast słowo ,,FizzBuzz'' w przypadku podzielności przez 15.

    \begin{verbatim}

      // Groovy:
      for (i in 1..100) {
        println "${i%3?'':'Fizz'}${i%5?'':'Buzz'}" ?: i
      }    

      //Java:
      public class FizzBuzz{
        public static void main(String[] args){
          for(int i= 1; i <= 100; i++){
            if(i % 15 == 0){
              System.out.println("FizzBuzz");
            } else if(i % 3 == 0){
              System.out.println("Fizz");
            } else if(i % 5 == 0){
              System.out.println("Buzz");
            } else{
              System.out.println(i);
            }
          }
        }
      }

    \end{verbatim}

    W szczególności filozofia Grails jest bardzo zbliżona do bardzo popularnego frameworka Ruby On Rails (RoR) umożliwiającego ekspresowe prototypowanie aplikacji umożliwiających (1) tworzenie, (2) odczyt, (3) aktualizację i (4) usuwanie danych za pomocą formularzy (popularnie zwanych ,,CRUD'' - create, retrieve, update, delete). Wiele przyjętych rozwiązań w Grails zostało zaczerpniętych z RoR, jednak oba frameworki znacznie różnią się pod względem technologicznym. Szczegółowe porównanie tych technologii leży poza zakresem tej pracy, jednak osoby zainteresowane, powinny zapoznać się z doskonałym wątkiem w serwisie Stackoverflow, przytoczonym w bibliografii \cite{RailsGr}.


    Grails jest silnie oparte o paradygmat ,,convention over configuration'' - architekturze zakładającej minimalizację potrzeby konfiguracji aplikacji przez programistę na rzecz przyjętych konwencji, do których należy się stosować, aby wydaje dostarczać działające rozwiązania. Podejście to dotyczy wszelkich aspektów z tworzeniem aplikacji w Grails - od struktury katalogów, przez umiejscowienie i zawartość plików konfiguracyjnych, po nazewnictwo klas, zmiennych i zastosowanie odpowiednich struktur danych. 

    Oparcie technologii Grails na wzorcu Model-View-Controller (MVC) jest naturalnym podejściem w nowoczesnych frameworkach ułatwiających tworzenie aplikacji internetowych.

    Struktura katalogów nowo utworzonej aplikacji wygląda następująco:

    \begin{verbatim}

      + grails-app
        + conf           ---> lokalizacja artefaktów konfiguracji
          + hibernate    ---> opcjonalne pliki konfiguracyjne hibernate
          + spring       ---> opcjonalne pliki konfiguracyjne spring 
        + controllers    ---> kontrolery aplikacji
        + domain         ---> klasy reprezentujące modele
        + i18n           ---> zlokalizowane komunikaty il8n
        + services       ---> warstwa serwisowa
        + taglib         ---> biblioteki znaczników
        + util           ---> klasy pomocnicze special utility classes 
        + views          ---> lokalizacja widoków
          + layouts      ---> lokalizacja 
        + lib
        + scripts       ---> scripts
        + src
        + groovy       ---> optional; location for Groovy source files
                              (of types other than those in grails-app/*)
        + java      ---> optional; location for Java source files
        + test          ---> generated test classes
        + web-app
          + WEB-INF

    \end{verbatim}

    Po wstępnej analizie, mając wiedzę na temat dziedziny problemu, w prosty sposób można utworzyć podstawową funkcjonalność aplikacji przykładowej aplikacji. Proces tworzenia elementów budowanego systemu rozpoczyna się od stworzenia klasy Groovy

    Dzięki systemowi wtyczek dostarczanej wraz z technologią Grails istnieje możliwość stosunkowo łatwej rozbudowy systemu o nowe funkcjonalności.


    \subsection{Spring Framework i Hibernate ORM}

      Sercem Grails jest Spring Framework oraz Hibernate ORM. Obie technologie są wiodące na rynku i szeroko stosowane w komercyjnych projektach dowolnych rozmiarów. Spring jest de facto zestawem narzędzi, wzorców projektowych i bibliotek realizujących ogromną ilość funkcjonalności, mających za zadanie przyśpieszyć proces wytwarzania oprogramowania oraz w pewnym zakresie zapewnić wysoką jakość tworzonych rozwiązań. Przede wszystkim jednak, jest tzw. kontenerem Inversion Of Control (odwrócenie sterowania) \cite{MFow01}. ********** tu trochę o IoC ***********. Prócz IoC Spring Framework to potężna platforma integrująca wiele rozwiązań, takich jak webowy framework Spring MVC, biblioteki i wzorce dotyczące bezpieczeństwa (Spring Security), wrapper jdbc, transakcje (Spring JDBC Templates), etc.

      Hibernate ORM jest technologią pozwalającą na mapowanie rekordów z relacyjnych baz danych, na obiekty w systemie. Jest to metoda w pewnym stopniu rozwiązująca problem niezgodności impedancji \cite{KSub01}

  \section{Pozostałe technologie}
    \subsection{Postgresql}
      Trwałość danych w aplikacjach internetowych standardowo realizowana jest przez bazę danych. Na potrzeby prezentowanego prototypu, wykorzystano bazę danych Postgresql. Jest to obiektowo-relacyjna baza danych udostępniona na licencji Wolnego i Otwartego Oprogramowania. Baza danych PostgreSQL implementuje znaczną część standardu SQL:2008 [!ref - http://www.postgresql.org/docs/9.1/static/features.html]

    \subsection{Javascript, jQuery i biblioteka jsUml2}
      
       
    \subsection{System kontroli wersji git i serwer heroku}
    \subsection{Srodowisko programistyczne (linux, vim, tmux)}

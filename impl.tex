\chapter{Rozwiązania implementacyjne}
  
  W poprzednim rozdziale opisano narzędzia jakie wykorzystano w trakcie implementacji prototypu proponowanego rozwiązania. W niniejszym rozdziale, zostaną zaprezentowane konkretne rozwiązania implementacyjne zastosowane w prototypie. Szczególną uwagę zwrócono na moduł edycji diagramów przypadków użycia oraz generatora specyfikacji wymagań.

  \section{Architektura systemu}
    Jak wspomniano w ogólnym opisie prototypu w rozdziale 4, definiowanie wymagań w systemie Reqmanager polega na jednoczesnym dostępie do edytora tekstu oraz edytora diagramów przypadków użycia. Użytkownik, mając otwartą aplikację, oprócz tekstowego opisu wymagania, ma możliwość dodawania oraz edycji przypadków użycia w formie graficznej.

    Aplikacja Reqmanager jest typową aplikacją internetową typu klient-serwer, gdzie klientem jest przeglądarka użytkownika korzystającego z aplikacji zainstalowanej na zdalnym serwerze, odpowiednio reagującej na żądania klienta na zasadzie żądanie - odpowiedź (request - response). Przeglądarka internetowa jest tak zwanym ,,cienkim klientem'' z racji silnego uzależnienia działania od serwera aplikacji. 
    
    Framework Grails narzuca architekturę kodu aplikacji według wzorca Model-View-Controller, opisanego w rozdziale 5.  

    Na potrzeby aplikacji Reqmanager, utworzono następującą strukturę modeli aplikacji:

    \begin{verbatim}
    Diagram 
    Project
    Requirement
    UseCase
    \end{verbatim}

    Struktura bazy danych jest generowana automatycznie, na podstawie analizy klas modeli:

    \begin{verbatim}
                    List of relations
    Schema |         Name          |   Type   |  Owner
    --------+-----------------------+----------+----------
    public | diagram               | table    | postgres
    public | hibernate_sequence    | sequence | postgres
    public | project               | table    | postgres
    public | requirement           | table    | postgres
    public | requirement_use_cases | table    | postgres
    public | use_case              | table    | postgres
    \end{verbatim}


    ************* TUTAJ STRUKTURA ERD SYSTEMU **************

    \subsection{Tworzenie diagramów}

      Diagramy przypadków użycia w systemie edytowane są po stronie klienta, dzięki wykorzystaniu elementu \emph{canvas} wprowadzonego niedawno do specyfikacji języka HTML5. Obsługa tej funkcjonalności po stronie przeglądarki, stanowi problem związany z zachowaniem stanu diagramu w systemie. W celu rozwiązania tego problemu, zaimplementowano mechanizm serializacji całego diagramu w formacie xml do bazy danych. Cel ten osiągnięto, wykorzystując metodę \emph{getXMLString()} udostępnioną przez wszystkie obiekty dziedziczące po klasie \emph{Diagram} w bibliotece jsUML2. Przykładową strukturę zserializowanego diagramu, umieszczno na Listingu \ref{lst:diagramXml}. 
      Dodanie nowego diagramu wiąże się zatem z inicjalizacją i wyświetleniem użytkownikowi obszaru roboczego diagramu bez komunikacji z serwerem. Diagram tworzony jest po załadowaniu widoku edycji wymagania:

      \begin{lstlisting}[caption={widok requriement/edit.gsp}, label={lst:appUseCase}]
      window.onload = function() {
        var app = new AppUseCase("ud_diagram_div");
      }

      \end{lstlisting}

      AppUseCase jest obiektem javascript typu singleton, którego skrócona implementacja została przedstawiona na poniższym listingu: 
    
      \begin{lstlisting}[caption={implementacja obiektu JS AppUseCase}, label={lst:appUseCaseImpl}]
      var AppUseCase = function(elementId) { 
        var xmlstr = document.getElementById('diagramXml').value
        var ucDiag = new UMLUseCaseDiagram() 
        if(xmlstr) {
          ucDiag.setXMLString(xmlstr);
        }

        var c1 = document.getElementById('c1');
        var c = c1.getContext('2d');
        var c2 = document.getElementById('c2');
        c2.onmousedown = function() { return false; };
        var mc = c2.getContext('2d')}

        var div = document.getElementById('ud_diagram_div');

        ucDiag.initialize(11, div, c, mc, 600, 1000);
        ucDiag.draw();
      }
      \end{lstlisting}

      Powyższy kod w javascript, pobiera wartość z elementu formularza i identyfikatorze ,,diagramXml'' oraz tworzy nowy obiekt UMLUseCaseDiagram, zdefiniowany w pakiecie \emph{modules} biblioteki jsUML2. Następnie weryfikuje czy pole \emph{diagramXml} jest niepuste. Jeśli warunek jest prawdziwy, wywołuje metodę \emph{setXMLString()} na obiekcie diagramu, dzięki czemu diagram inicjalizowany jest z wartościami wcześniej zserializowanymi do formatu xml i zapisanymi w bazie danych. Kolejne wiersze zajmują się utworzeniem kontekstów na bazie elementów \emph{canvas}, aby na końcu zainicjować i narysować odpowiedni diagram w oknie przeglądarki użytkownika. Należy jednocześnie zaznaczyć, iż w przypadku gdy nie istnieje zserializowana wersja diagramu w bazie (pole \emph{diagramXml} jest puste), wówczas zostanie zainicjalizowany nowy, pusty diagram, nieposiadających żadnych elementów. Pole \emph{diagramXml} jest wcześniej ustawiane na podstawie danych z bazy.

      \subsubsection{Zapis danych xml}

      Dopiero gdy użytkownik wybierze opcję zapisu aktualnego wymagania, zserializowany stan diagramu w formacie xml, zostanie przesłany metodą POST do serwera. Następnie otrzymany po stronie serwera ciąg znaków xml zostanie przetworzony, utworzony zostanie obiekt klasy Diagram i na końću nastąpi zapis obiektu do bazy danych. Za ten proces odpowiedzialna jest akcja \emph{save} kontrolera \emph{Requirement}. Jest to dość ograniczone rozwiązanie, które można byłoby usprawnić wprowadzając asynchroniczny, automatyczny zapis stanu diagramu co kilka sekund lub po każdej aktualizacji po stronie klienta.


    \begin{lstlisting}[caption={struktura xml diagramu}, label={lst:diagramXml}]

      <UMLUseCaseDiagram name="Use case diagram">
        <UMLSystem id="UMLSystem_4" x="165" y="97" ... >
          <superitem id="stereotypes" visibleSubComponents="true"/>
          <item id="name" value="System name"/>
          <UMLUseCase id="UMLUseCase_1" x="232" y="142" ... >
            <superitem id="stereotypes" visibleSubComponents="true"/>
            <item id="name" value="test test"/>
          </UMLUseCase>
          <UMLUseCase id="UMLUseCase_3" x="222" y="209" ... >
            <superitem id="stereotypes" visibleSubComponents="true"/>
            <item id="name" value="for req 74"/>
          </UMLUseCase>
        </UMLSystem>
      </UMLUseCaseDiagram>

    \end{lstlisting}

      Baza danych postgresql obsługuje specjalny typ danych ,,xml''. Dzięki obsłudze tego typu danych, po stronie bazy dokonywana jest weryfikacja poprawnosci dokumentu jaki programista ma zamiar zapisać. Framework Grails nie współpracuje dobrze z typem danych xml w bazie, dlatego niezbędna była implementacja rozwiązania rzutującego typ String na typ SQLXMLType.W tym celu została stworzona specjalna klasa w Javie, implementująca interfejs \emph{org.hibernate.usertype.UserType}. Hibernate udostępnia ten interfejs umożliwiając programistom tworzenie własnych typów danych i zdefiniowanie ich zachowania w stosunku do bazy danych. Dodatkowo, niezbędne było odpowiednie przygotowanie modelu, podając typ pola przechowującego xmla jako String, a następnie sworzyć odpowiedni blok ,,mapping'' w modelu. Na Listingu \ref{lst:xml}, dla przykładu zaprezentowano implementację klasy Diagram z systemu Reqmanager:

      \begin{lstlisting}[caption={implementacja modelu Diagram}, label={lst:xml}]
      package pl.edu.pjwstk.reqmanager
      public class Diagram {
        String name
        String xmlString

        static belongsTo = Requirement
        static mapping = {
          xmlString type: SQLXMLType
        }
        static constraints = {
          name(nullable:true)
        }
      }
      \end{lstlisting}
      
      \begin{lstlisting}[caption={implementacja niestandardowego typu SQLXMLType}, label={lst:xmlType}]
      package pl.edu.pjwstk.reqmanager
       
      import java.io.Serializable;
      import java.sql.PreparedStatement;
      import java.sql.ResultSet;
      import java.sql.SQLException;
      import java.sql.Types;
      import org.hibernate.HibernateException;
       
      /**
       * Store and retrieve a PostgreSQL "xml" column as a Java string.
       */
      public class SQLXMLType implements org.hibernate.usertype.UserType {
       
          private final int[] sqlTypesSupported = new int[] { Types.VARCHAR };
       
          @Override
          public int[] sqlTypes() {
              return sqlTypesSupported;
          }
       
          @Override
          public Class returnedClass() {
              return String.class;
          }
       
          @Override
          public boolean equals(Object x, Object y) throws HibernateException {
              if (x == null) {
                  return y == null;
              } else {
                  return x.equals(y);
              }
          }
       
          @Override
          public int hashCode(Object x) throws HibernateException {
              return x == null ? null : x.hashCode();
          }
       
          @Override
          public Object nullSafeGet(ResultSet rs, String[] names, Object owner) throws HibernateException, SQLException {
              assert(names.length == 1);
              String xmldoc = rs.getString( names[0] );
              return rs.wasNull() ? null : xmldoc;
          }
       
          @Override
          public void nullSafeSet(PreparedStatement st, Object value, int index) throws HibernateException, SQLException {
              if (value == null) {
                  st.setNull(index, Types.OTHER);
              } else {
                  st.setObject(index, value, Types.OTHER);
              }
          }
       
          @Override
          public Object deepCopy(Object value) throws HibernateException {
              if (value == null)
                  return null;
              return new String( (String)value );
          }
       
          @Override
          public boolean isMutable() {
              return false;
          }
       
          @Override
          public Serializable disassemble(Object value) throws HibernateException {
              return (String) value;
          }
       
          @Override
          public Object assemble(Serializable cached, Object owner) throws HibernateException {
              return (String) cached;
          }
       
          @Override
          public Object replace(Object original, Object target, Object owner) throws HibernateException {
              return original;
          }
      }
      \end{lstlisting}

    \subsection{Współdzielenie diagramów}

    Istotnym elementem systemu jest współdzielenie diagramów między wieloma wymaganiami. Każdy diagram może zostać przypisany do kilku wymagań, dzięki czemu wiele wymagań może wykorzystywać i rozszerzać już istniejące diagramy. Modyfikacja diagramu przypadku użycia w trakcie edycji konkretnego wymagania, wiąże się z aktualizacją stanu diagramu we wszystkich korzystających z niego wymaganiach.
    
    \subsection{Mapowanie przypadków użycia}
      W związku z możliwością współdzielenia diagramów pomiędzu wieloma wymaganiami, w aktualnej wersji prototypu, przypadki użycia są łączone z wymaganiem, tylko po wykonaniu przez użytkownika akcji wskazania przypadku użycia jako realizującego aktualnie modyfikowane wymaganie.

      Na etapie edycji diagramu, użytkownik ma możliwość wskazania konkretnych use case'ów znajdujących się na diagramie, jako odpowiedzialnych za realizację aktualnie edytowanego wymagania. Powiązanie przypadku użycia z wymaganiem polega na wyborze opcji \emph{mark for requirement} z paska narzędzi diagramu, a następnie kliknięcie w dany przypadek użycia na diagramie. Po zatwierdzeniu operacji, zostaje wysłane asynchroniczne żądanie do serwera, pod adres http://reqmanager.herokuapp.com/requirement/addUseCase/id\_wymagania. W żądaniu przekazywane są do serwera parametry niezbędne dla zestawienia przypadku użycia z wymaganiem. Poniżej zamieszczono listę tych parametrów:

    \begin{itemize}
      \item \emph{id} - identyfikator wymagania do którego dodawany jest przypadek użycia
      \item \emph{clickedUCName} - nazwa przypadku użycia wprowadzona przez użytkownika
      \item \emph{clickedUCId} - identyfikator przypadku użycia wygenerowany automatycznie przez bibliotekę jsUML2
    \end{itemize}

    Asynchronicznie wysłane do serwera żądanie zostaje przekazane wraz z paramterami do akcji \emph{addUseCase} kontrolera \emph{Requirement}. Kod odpowiedzialny za połączenie przypadku użycia z wymaganiem, został załączony na Listingu \ref{lst:addUseCase}.

    \begin{lstlisting}[caption={addUseCase}, label={lst:addUseCase}]
    def addUseCase = {
      def requirement = Requirement.get(params.id)
      def useCase = UseCase.findByTitle(params.clickedUCName)

      if(useCase == null) {
        useCase = new UseCase(title: params.clickedUCName, code: params.clickedUCId)
          requirement.addToUseCases(useCase)
          render(text: "ok")
      } else {
        if(!requirement.useCases.contains(useCase)) {
          requirement.addToUseCases(useCase)
          render(text: "ok")
        } else {
          render(text: "wymaganie zawiera ten przypadek")
        }
      }
    }
    \end{lstlisting}

    Jak można odczytać z powyższego kodu źródłowego, akcja ta, pobiera odpowiednie wymaganie oraz podejmuje próbę znalezienia przypadku użycia po nazwie. Jeśli dany przypadek użycia nie istnieje jeszcze w bazie danych (nowo utworzony przypadek użycia), zostaje swtorzony nowy obiekt klasy UseCase z odpowiednimi parametrami (nazwa nadana przez użytkownika oraz identyfikator wygenerowany po stronie klienta). W przeciwnym razie, zostaje on dodany do listy przypadków użycia połączonych asocjacją z danym wymaganiem, o ile nie istnieje już na tej liście. Cała operacja wykonywana jest asynchronicznie, po stronie serwera. 

    \subsection{Przypadki użycia a wymagania}
    
      W celu odpowiedniego oznaczenia przypadków użycia przypisanych do wybranego wymagania, zaimplementowano algorytm dwustopniowej inicjalizacji diagramu. Pierwszym krokiem jest ekstrakcja z bazy danych odpowiedniego wymagania, weryfikacja czy do danego wymagania przypisany jest diagram oraz czy istnieją jakieś przypadki użycia dla wymagania. Na podstawie informacji z bazy danych przekazanych przez kontroler, ładowany jest widok \emph{requirement/show.gsp}. Następnie, po załadowaniu większej części strony uruchamiany jest skrypt po stronie klienta, odwołujący się asynchronicznie z powrotem do serwera z ,,pytaniem'' o nazwy przypadków użycia przypisanych do danego wymagania. Jednocześnie, do pamięci ładowana jest struktura xml diagramu. Program iteruje po wszyskich węzłach w poszukiwaniu pokrywających się przypadków użycia. Jeśli znajdzie choć jeden, zmienia kolor jego tła na żółty.
      
    
    \begin{lstlisting}[caption={Kolorowanie tła przypadków użycia}, label={lst:colorUC}]

    var reqUseCases = $.ajax({
      type: "GET",
      url: "http://reqmanager.herokuapp.com/requirement/getUseCases/" + reqId,
      cache: false
    }).done(function(xht){ 
      var dom = (new DOMParser()).parseFromString(diag.diagram.getXMLString(), "text/xml");
      var documentEle = dom.documentElement;
      var usecaseArray = documentEle.getElementsByTagName('UMLUseCase');

      for(var i = 0; i < usecaseArray.length; i++) {
        for(var j = 0; j < usecaseArray[i].childNodes.length; j++) {
          var itemValue = usecaseArray[i].childNodes[j].attributes.value;
          if(itemValue != undefined) {
            if(xht.indexOf(itemValue.value.toString()) != -1) {
              var divv = document.getElementById("ud_diagram_div");
              var foundUseCase = null;

              for(var k = 0; k < diag.diagram._nodes.length; k++) {
                if(diag.diagram._nodes[k].getName().toString() 
                  === xht[xht.indexOf(itemValue.value.toString())]) {
                  foundUseCase = diag.diagram._nodes[k]; 
                }
              };
              if(foundUseCase) foundUseCase.setBackgroundColor("#F3F5BC");
              diag.diagram.draw();
            }
          }
        }
      }
    });
    \end{lstlisting}

    \subsection{Edytor tekstu}
      W celu ułatwienia użytkownikowi pracy z tekstem, zaimplementowano edytor, oparty na bibliotece Ace Editor. 
      Kod inicjujący edytor tekstu, znajduje się poniżej:

    \begin{lstlisting}[caption={Inicjalizacja edytora tekstu}, label={lst:colorUC}]
    var editor = ace.edit("description");
    var MarkdownMode = require("ace/mode/markdown").Mode;
    editor.setTheme("ace/theme/clouds_midnight");
    editor.getSession().setMode(new MarkdownMode());
    editor.getSession().setUseWrapMode(true);
    \end{lstlisting}

    Powyższy kod inicjuje edytor, załączając go do wskazanego węzła w dokumencie html (w tym przypadku jest to warstwa ,,description''). Następnie ustawia wygląd graficzny edytora oraz przestawia go w tryb pracy z formatem Markdown.
    Zawartość edytora tekstu, przekazywana jest do serwera po wysłaniu formularza przez użytkownika. Ponieważ zawartość formularza znajduje się w elemencie \emph{div} w strukturze DOM strony, w celu przekazania jego wartości, jest ona kopiowana do specjalnego pola typu textarea:

    \begin{lstlisting}[caption={Inicjalizacja edytora tekstu}, label={lst:colorUC}]
    $("#submitbtn").click(function() {
      textarea.val(editor.getSession().getValue()); 
      $("#diagramXml").val(app.diagram.getXMLString());
    });
    \end{lstlisting} 

    \subsection{Generowanie dokumentacji}

  \section{Zalety i wady proponowanego rozwiązania}
    Proponowane rozwiązanie jest podjęciem próby uproszczenia procesów inżynierii wymagań. Głównym celem stworzenia prototypu było dostarczenie narzędzia umożliwiającego efektywne zbieranie i przetwarzanie wymagań przy jednoczesnym uproszczeniu poziomu jego skomplikowania. Powstał prototyp realizujący te cele, jednak siłą rzeczy jest to narzędzie przystosowane jedynie do małych projektów. W przeciwieństwie do istniejących rozwiązań, filozofia proponowanego systemu inspirowana była filozofią systemów Unix - programy i aplikacje powinny wykonywać jedną czynność i powinny ją wykonywać możliwie najlepiej. Jednocześnie należy tworzyć aplikacje tak, aby mogły w łatwy sposób współpracować z innymi systemami. Dlatego proponowane rozwiązanie, aby rozwiązywało skutecznie problemy ogólnie pojętej inżynierii oprogramowania, powinno być stosowane komplementarnie z innymi narzędziami, w szczególności systemem zarządzania projektami i zespołem. Od świadomości i dojrzałości organizacji zależy, jak taka infrastruktura zostanie zbudowana oraz jak wybrane narzędzia będą funkcjonowały w organizacji.

    Dużą zaletą systemu jest dostępność przez przeglądarkę internetową przy jednoczesnej realizacji funkcjonalności manipulacji obiektami graficznymi. W szczególności zastosowanie technologii HTML5 daje przewagę w dostępności nad istniejącymi rozwiązaniami, najczęściej implementowanymi w technologii Flash. Rzadko też można spotkać podobny poziom elastyczności aplikacji do zarządzania wymaganiami. Koncepcja systemu celowo nie narzuca wielu ograniczeń związanych ze strukturą wymagań w wierze, że brak restrykcji projektanta wpłynie pozytywnie na proces definiowania i formułowania wymagań. 

    Sam wybór technologii jaką jest Grails, jest zaletą stworzonego prototypu. Zastosowanie tej technologii umożliwia łatwą rozbudowę i skalowanie systemu. System wtyczek platformy Grails sprzyja modułowej naturze aplikacji stworzonych w tej technologii. Z kolei zarządzający całą aplikacją ,,silnik'' oparty na String Framework, Hibernate ORM oraz filozofii ,,convention over configuration'' sprzyjają stosowaniu najlepszych praktyk programistycznych i ponowne wykorzystanie kodu.

    Z pewnością istotną wadą jest brak obsługi ,,śledzenia'' wymagań (traceability). Jednak w istniejących rozwiązaniach śledzenie wymagań nie rozwiązuje w pełni problemu. Nadal należy dbać o zgodność kodów i identyfikatorów wymagań w komunikacji z wieloma systemami. W dobrze zorganizowanej firmie, te problemy można wyeliminować przez narzucenie odpowiedniej kultury pracy, dbałość o szczegóły i skrupulatność w raportowaniu implementowanych rozwiązań do różnych systemów, jednak dopóki istnieje wiele narzędzi wspomagających różne etapy w procesie wytwórczym, odpowiednia komunikacja i wymiana danych jest jedyną metodą umożliwiającą skuteczne śledzenie wymagań.

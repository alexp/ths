\section{Streszczenie}

Brak ujednoliconego środowiska dokumentowania wymagań użytkownika to jeden z głównych problemów małych i średnich firm, w trakcie fazy analizy. Proces dokumentowania wymagań użytkownika zdaje się być zaniedbanym aspektem zarządzania projektami na rynku nowoczesnych aplikacji internetowych. Istniejące rozwiązania są drogie i często przystosowane do dużych projektów. W niniejszej pracy zaprezentowano narzędzie ułatwiające proces zbierania i przetwazania wymagań użytkownika z wykorzystaniem technologii internetowych, w niewielkich projektach. 

W niniejszej pracy, zamiast skupiać się na powielaniu i rozszerzaniu funkcjonalności istniejących systemów zarządzania wymaganiami, zaprezentowano rozwiązanie dla małych i średnich projektów, pozwalające skupić się na najistotniejszych akpektach formułowania  i dokumentowania wymagań. Powstały system umożliwia wprowadzanie i kategoryzowanie wymagań zarówno w formie graficznej, jak i tekstowej. Na podstawie wprwadzonych danych, aplikacja umożliwia automatyczną generację dokumentacji wymagań użytkownika. 

% to chyba zbyt szczegolowe
Do opisu wymagań w formie tekstowej, system udostępnia edytor tekstowy wspierający podświetlanie składni wprowadzanego tekstu w formacie Markdown. Zastosowanie prostego w użyciu formatu sprawia, że użytkownik może w pełni skupić się na konstrukcji treści opisu wymagania. Jednocześnie zachowana zostaje struktura dokumentu oraz wizualny podział na odpowiednie sekcje. // 

Dzięki zastosowaniu powszechnie dostępnych  technologii internetowych, proponowane narzędzie może zostać uruchomione w wielu różnych środowiskach. Wykorzystanie technologii  HTML5 oraz języka javascript umożliwiło stworzenie narzędzia wspomagającego gaficzne modelowanie przypadków użycia, będących bazą do ostatecznego wygenerowania dokumentacji wymagan uzytkownika. 

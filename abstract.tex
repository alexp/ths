\begin{abstract}

  Praca dotyczy zagadnienia zbierania wymagań i narzędzi związanych z tym procesem. Brak ujednoliconego środowiska dokumentowania wymagań to jeden z głównych problemów małych i średnich firm, w trakcie fazy analizy. Procesy inżynierii wymagań zdają się być zaniedbanym aspektem zarządzania projektami na rynku nowoczesnych aplikacji internetowych a istniejące rozwiązania są drogie i często przystosowane do projektów o dużej skali. 

  W niniejszej pracy zaprezentowano rozwiązanie dla małych i średnich projektów. Pozwala ono skupić się na najważniejszych aspektach formułowania i dokumentowania wymagań oraz umożliwia wprowadzanie i kategoryzowanie zebranej wiedzy zarówno w formie graficznej, jak i tekstowej. Na podstawie zgromadzonych danych, aplikacja umożliwia automatyczną generację dokumentu specyfikacji wymagań. 

  Dzięki zastosowaniu powszechnie dostępnych technologii internetowych, proponowane narzędzie może zostać uruchomione w wielu różnych środowiskach. Wykorzystanie technologii HTML5 oraz języka JavaScript umożliwiło stworzenie narzędzia wspomagającego graficzne modelowanie przypadków użycia.

\end{abstract}

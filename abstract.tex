\begin{abstract}

  Brak ujednoliconego środowiska dokumentowania wymagań użytkownika to jeden z głównych problemów małych i średnich firm, w trakcie fazy analizy. Proces dokumentowania wymagań użytkownika zdaje się być zaniedbanym aspektem zarządzania projektami na rynku nowoczesnych aplikacji internetowych. Istniejące rozwiązania są drogie i często przystosowane do dużych projektów. W niniejszej pracy zaprezentowano narzędzie ułatwiające proces zbierania i przetwazania wymagań użytkownika z wykorzystaniem technologii internetowych, w niewielkich projektach. 

  W niniejszej pracy zaprezentowano rozwiązanie dla małych i średnich projektów, pozwalające skupić się na najważniejszych akpektach formułowania i dokumentowania wymagań. Powstały system umożliwia wprowadzanie i kategoryzowanie wymagań zarówno w formie graficznej, jak i tekstowej. Na podstawie wprowadzonych danych, aplikacja umożliwia automatyczną generację dokumentu specyfikacji wymagań. 

  Dzięki zastosowaniu powszechnie dostępnych technologii internetowych, proponowane narzędzie może zostać uruchomione w wielu różnych środowiskach. Wykorzystanie technologii HTML5 oraz języka javascript umożliwiło stworzenie narzędzia wspomagającego graficzne modelowanie przypadków użycia.

\end{abstract}

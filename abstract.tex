\begin{abstract}

  Praca dotyczy zagadnienia zbierania i dokumentowania wymagań systemów informatycznych w małych i średnich przedsiębiorstwach oraz narzędzi związanych z tym procesem. Procesy inżynierii wymagań zdają się być zaniedbanym aspektem zarządzania projektami na rynku nowoczesnych aplikacji internetowych a istniejące rozwiązania są drogie i często przystosowane do zastosowań o dużej skali. W niniejszej pracy podjęto próbę wypracowania rozwiązania mającego zastosowanie w niewielkich, dynamicznych projektach. Proponowana koncepcja pozwala skupić się na najważniejszych aspektach formułowania i dokumentowania wymagań oraz umożliwia wprowadzanie i kategoryzowanie zebranej wiedzy. Na podstawie zgromadzonych danych, aplikacja umożliwia automatyczną generację dokumentu specyfikacji wymagań. Technologia HTML5 w połączeniu z językiem JavaScript umożliwiła stworzenie narzędzia wspomagającego graficzne modelowanie przypadków użycia, które może zostać uruchomione w wielu różnych środowiskach.

\end{abstract}

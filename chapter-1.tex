\chapter{Wstęp}

    Rozwój technologii internetowych w ostatnich latach umożliwił programistom tworzenie wysoce interaktywnych narzędzi przy jednoczenym ograniczeniu wymagań systemowych jakie muszą spełnić klienci, chcący skorzystać z wytworzonego oprogramowania. W szczególności, niedawno wprowadzona na rynek, technologia HTML5 daje nowe, szerokie możliwości przeglądarkom internetowym. Możliwości, które były dotychczas zarezerwowane głównie dla technologii Flash (oraz z trochę gorszymi wynikami MS Silverlight, JavaFX, itp.) stają się wspierane natywnie, w każdej popularnej przeglądarce internetowej. 

    Z wymaganiami użytkownika stykamy się praktycznie od samego początku pracy nad projektem informatycznym. Z tego powodu potrzebne są skuteczne narzędzia gromadzenia i przetwarzania wymagań użytkownika oraz komunikowania ich wszystkim członkom zespołu. Dzięki wyżej wspomnianej technologii HTML5, istnieje możliwość usprawnienia procesu dokumentowania wymagań przy jednoczesnym zachowaniu wysokiej dostępności aplikacji, włączając do niego nowe narzędzia oparte na metodach graficznych. 

    Rynek nowych technologii roi się dzisiaj od aplikacji rozwiązujących rozmaite problemy, od prostych list zakupów, po zaawansowane narzędzia wspierające zarządzanie projektami. Pomimo istnienia wielu aplikacji skupiających się na procesach fazy analizy, bardzo ciężko znaleźć rozwiązanie dostosowane do potrzeb i realiów prowadzenia projektów w małych i średnich przedsiębiorstwach. 
          
    W niniejszej pracy zostanie zaprezentowana koncepcja aplikacji wspierającej zarządzanie wymaganiami użytkownika, stworzona z myślą o rzeczywistych problemach jakie wiążą się z tym etapem projektu informatycznego. Zaproponowany system ma za zadanie ułatwić zespołom projektowym komunikację i dostęp do wiedzy przy jednoczesnym skupieniu uwagi na najistotniejszych aspektach definiowania i przetwarzania wymagań. ~\cite{Leite87}

    % check
    W niniejszej pracy zaprezentowano nowoczesne narzędzie wspierające proces zbierania wymagań użytkownika. Przyjęte rozwiązania mają za zadanie ułatwić zespołom projektowym komunikację, dostęp do wiedzy oraz wizualizację przypadków użycia w kontekście fazy analizy i zbierania wymagań. Głównymi założeniami przy tworzeniu koncepcji rozwiązania, były m.in. prostota obsługi oraz skupienie uwagi użytkownika na najważniejszych aspektach definiowaVnia wymagania. 


    \section{Cel pracy}

      W niewielkich projektach, proces gromadzenia i dokumentowania wymagań użytkowników jest najczęściej słabo sformalizowany. Wymagania trafiają do zespołu z różnych, heterogenicznych źródeł i w wielu, zwykle niekompatybilnych, formatach. Pomimo faktu, iż na rynku nie brakuje oprogramowania wspierającego zarządzanie wymaganiami, istniejące rozwiązania, często nie są przystosowane do realiów prowadzenia projektów w małych, dynamicznych przedsiębiorstwach. Skomplikowane i trudno dostępne systemy często wymagają instalacji oprogramowania po stronie klienta [!ref] a rozwiązania dostępne online, w formule SaaS (Sofware As A Service) [!ref] często powielają tylko funkcjonalnosći potężnych (i drogich) systemów takich jak IBM DOORS [!ref] czy IBM ReqiusitePro [!ref]. Ponadto niezwykle trudno jest znaleźć rozwiązanie darmowe lub posiadające ogólno dostępną wersję demonstracyjną. 

      Celem niniejszej pracy jest konstrukcja prototypu systemu usprawniającego zbieranie i przetwarzanie wymagań użytkownika w postaci tekstowej oraz graficznej, dostarczając narzędzi edycji tekstu oraz graficznego modelowania przypadków użycia. W efekcie, użytkownik, po wprowadzeniu początkowych wymagań, ma możliwość automatycznego wygenerowania dokumentu SRS (Software Requirement Specification) [!ref]. Podejście zaprezentowane w tej pracy, opiera się na założeniach, że nowoczesne oprogramowanie wspierające zarządzanie wymaganiami, powinno m.in.: stawnowić centralne repozytorium wiedzy o wymaganiach; być łatwo dostępne w jak największej ilości różnych środowisk; umożliwiać łatwą kolaborację oraz być łatwe w obsłudze, prowokując do kreatywności, zamiast stawiać bariery w postaci skomplikowanych formularzy i tabel. 


    \section{Zarys koncepcji}

      Proces zbierania wymagań jest trudny, ponieważ często jest procesem niesformalizowanym, rozproszonym i udokumentowanym na wiele sposobów (różne nośniki danych, niekomatybilne oprogramowanie). Ponadto wymagania zwykle pochodzą od wielu interesariuszy: od sponsora projektu, po użytkowników końcowych. Niezależnie od poziomu dojrzałości organizacji i stosowanej metodyki zarządzania projektami, źródła pochodzenia wymagań pozostają rozproszone i niekompatybilne. 

      Na potrzeby ninejszej pracy, został stworzony prototyp systemu pozwalający w łatwy sposób dokumentować gromadzone wymagania użytkownika. W powstałym systemie, głownymi narzędziami definiowania wymagań są pliki tekstowe oraz diagramy przypadków użycia. Osoby odpowiedzialne w projekcie za zarządzanie wymaganiami, otrzymują proste w obsłudze narzędzia, pozwalające na skupienie się nad sednem poroblemu, który poszczególne wymaganie ma na celu rozwiązać. Zaimplementowano system zarządzania projektem, w którym użytkownik ma możliwość zdefiniowania podstawowych parametrów takich jak nazwa projektu, planowany czas zakończenia oraz ogólny opis, zawierający kluczowe informacje o projekcie na etapie analizy. Korzystając z narzędzia definiowania wymagań w projekcie, użytkownik ma do dyspozycji edytor tekstowy, w którym dokumentuje zidentyfikowane wymagania użytkownika. Dzięki wykorzystaniu prostego języka znaczników (Markdown [!ref]), użytkownik skupia się na logicznej strukturze opisu wymagania. Brak narzędzi "WYSIWYG" znanych ze standardowych edytorów tekstu sprawia, że użytkownik nie jest rozpraszany potrzebą myślenia o graficznej reprezentacji tekstu opisującego problem, przy jednoczesnym zachowaniu czytelnej struktury dokumentu. System załączników i komenatrzy, umożliwia iteracyjną pracę nad wymaganiami przy wykorzystaniu zewnętrzych źródeł informacji. 

      Dzięki graficznemu edytorowi przypadków użycia standardu UML, użytkownik ma możliwość definiowania kluczowych funkcjonalności systemu i łączenia ich z wybranymi wymaganiami. Tworzone przez użytkowników diagramy, są dostępne do ponownego wykorzystania w innych miejscach w systemie, dzięki czemu wspierana jest kolaboracja i korzystanie z już istniejących rozwiązań. Zarówno opisy wymagań jak i przypisane im przypadki użycia, są integralną częścią dokumentu specyfikacji wymagań użytkownika. Dlatego na każdym etapie fazy analizy istnieje możliwość automatycznego wygenerowania specyfikacji wymagań na podstawie danych wprowadzonych do systemu. Takie podejście sprzyja iteracyjnemu podejściu do tworzenia specyfikacji wymagań i umożliwia prezentację specyfikacji już na wczesnym etapie projektu. 

      Pozyskiwanie wymagań jest procesem kreatywnym. Koncepcja systemu bazuje na silnym przekonaniu, że brak ograniczeń w postaci skomplikowanych przytłaczających fukncjonalności pozwala skupić się na procesie twórczym w fazie definiowania wymagań, rozwiązujących rzeczywiste problemy. 

      % To jest trochę niebezpieczne, bo będzie trzeba tę hipotezę zweryfikować
      Głównym założeniem projektu, jest następująca hipoteza: dostarczenie odpowiednich, przemyślanych i prostych narzędzi, łatwych do natychmiastowego wdrożenia w projekcie, powoduje zmniejszenie czasu pracy nad specyfikacją wymagań użytkownika oraz zwiększa jakość powstałych w tym procesie wymagań. 

      Głównym założeniem projektu jest następująca hipoteza: ograniczenie możliwości definiowania formatu i struktury opisu wymagań, dostarczenie graficznych metod modelowania przypadków użycia oraz możliwość wygenerowania specyfikacji na podstawie danych wprowadzonych do systemu, ułatwi proces odkrywania i dokumentowania wymagań. Użytkownik nie powinien być jednocześnie przytłoczony ilością funkcjonalności dostępnych w aplikacji, a narzędzia którmi dysponuje powinny być intuicyjne i nie wymagać żadnej dokumentacji. 


    \section{Struktura pracy}

      W rozdziale 2 zostaną przedstawione, dostępne na rynku, narzędzia wspierające zarządzanie wymaganiami użytkownika. Rodział 3 jest dokładnym opisem dziedziny problemu oraz koncepcji zaimplementowanego prototypu. W rozdziale 4 zaprezentowano najistotniejsze technologie, jakie wykorzystano podczas pracy nad prototypem oraz krótkie omówienie środowiska programistycznego w jakim powstał przedmiot pracy. Rozdział 5 stanowi opis zaimplementowanego prototypu. W rozdziale 6 zawarto podsumowanie wyników pracy oraz propozycje kierunków dlaszego rozwoju. 

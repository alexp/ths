\chapter{Istniejące rozwiązania}

  W poprzednim rozdziale nakreślono temat przewodni niniejszej pracy. Poruszono problem dostępności i użyteczności istniejących narzędzi wspomagających zarządzanie wymaganiami. Zaprezentowano także autorską koncepcję rozwiązania, mającego na celu usprawnienie procesu tworzenia specyfikacji wymagań użytkownika. 

  W tym rozdziale zostanie przeprowadzona analiza rynku oprogramowania wspierającego fazę analizy. Druga część rozdziału zajmie się identyfikacją najistotniejszych problemów jakie posiadają istniejące rozwiązania. 

  \section{Analiza rynku systemów wsparcia zbierania wymagań}

    Na rynku nie brakuje systemów wsparcia procesu zbierania wymagań. Dostępne rozwiązania oferowane są w zasadzie w trzech różnych modelach. Najpopularniejszą ostatnio architekturą jest Software As A Service, czyli aplikacja internetowa zainstalowana na serwerach twórcy oprogramowania. Również klasyczne aplikacje okienkowe, które należy zainstalować na komputerze klienta nadal cieszą się dużą popularnością. Należy jednak zaznaczyć, iż w przeważającej większości są to starsze systemy, nierzadko implementowane jeszcze przed popularyzacją zaawansowanych aplikacji webowych. Niektóre firmy oferują także platformy w architekturze klient-serwer, wymagające istnienia zarówno centralnego serwera - repozytorium w sieci jak i desktopowych aplikacji klienckich. W tej sekcji zostaną przedstawione i porównane wybrane aplikacje z każdej z trzech kategorii. 

    \subsection{Model SaaS (Sowtware as a Service)}

      % Największym sukcesem w modelu SaaS zdaje się być platforma Force.com stworzona przez Salesforce (system klasy ERP).

      W ostatnim czasie, wraz z popularyzacją i rozwojem technologii internetowych znacznie wzrosły techniczne możliwości aplikacji dostępnych z poziomu przeglądarki internetowej. Tendencja ta sprzyja powstawaniu licznych aplikacji, adresujących bardzo specyficzne problemy, często w obrębie ściśle określonego segmentu rynku. Jednym z przykładów takiego wąskiego segmentu mogą być np. aplikacje do zarządzania projektami online. Wśród innych popularnych rozwiązań znajdują się takie produkty jak basecamp.com, polski nozbe.com czy unfuddle.com. Innym przykładem mogą być aplikacje wspierające proces rekrutacji (humanway.com, recruiterbox.com, jobvite.com). Na uwagę zasługują także aplikacje wspierające rezerwacje hoteli, niezależnych pokoi i mieszkań jak airbnb.com, booking.com czy b\&b.com. Naturalnie powyższe przykłady dotyczą tylko skrawka możliwych do zagospodarowania segmentów. W rzeczywistości, bardzo ciężko znaleźć wertykalny rynek, który jest niezagospodarowany serwisami, oferującymi różne podejścia do rozwiązania problemów danego segmentu. 

      Powodem takiego stanu rzeczy jest szeroko pojęta popularyzacja internetu oraz przenoszenie się do sieci firm tworzących opgrogramowanie. Dystrybucja oprogramowania w formule SaaS ma wiele zalet w stosunku do klasycznych aplikacji "desktopowych". W szczególności pomięty jest w tym przypadku proces rozprowadzania aplikacji do klientów za pomocą sieci stacjonarnych sklepów z oprogramowaniem. Zaimplementowane rozwiązanie, jest gotowe do użycia z chwilą udostępnienia w internecie. Udostępnienie aplikacji na serwerach twórcy oprogramowania daje nieograniczone możliwości wprowadzania nowych funkcjonalności i poprawek. Z kolei monitorowanie zachowań klientów pozwala niemalże w czasie rzeczywistym odpowiadać na potrzeby użytkowników. 

      W związku z licznymi zaletami modelu SaaS, w internecie powstało wiele aplikacji próbujących zaadresować problemy związane z procesem zbierania i dokumentowania wymagań użytkownika. Do najciekawszych rozwiązań można zaliczyć gatherspace (http://gatherspace.com/) [!ref] <<<<<<<< krótki opis głównych osobliwości gatherspace'a >>>>>>>>, tracecloud.com <<<<<<<< krótki opis głównych osobliwości tracecloud'a >>>>>>>>, accompa.com <<<<<<<< krótki opis głównych osobliwości accompy >>>>>>>> 

      \subsubsection{Gatherspace}
      \subsubsection{Tracecloud}
      \subsubsection{Accompa}

    \subsection{Aplikacje desktopowe}

      Model SaaS, mimo wszystkich swoich zalet, posiada również wady. Głównym powodem kontrowersji jest konieczność powierzenia danych biznesowych firmie udostępniającej narzędzie na swoich serwerach. Dla dużych korporacji, pilnie strzegących swoich tajemnic handlowych, takie rozwiązanie, może być nie do zaakceptowania ze względów bezpieczeństwa. Mimo potencjalnej możliwośći uniknięcia kosztów budowy i utrzymania infrastruktury sprzętowo-software'owej ryzyko związane z brakiem kontroli nad powierzonymi danymi jest często zbyt wielkie. 

      W klasycznych aplikacjach okienkowych, odpowiedzialność w zakresie zabezpieczenia danych spoczywa na samej korporacji i jej pracowniku. Dzięki temu, nadal istnieje zapotrzebowanie na oprogramowanie instalowane na lokalnym dysku użytkownika. Przykładami takich aplikacji są m.in. IBM DOORS, IBM Rational RequisitePRO oraz seapine.com. 

      \subsubsection{RequisitePRO}
      \subsubsection{DOORS}
      \subsubsection{seapine}

    \subsection{Platforma IBM .jazz}

      Platforma IBM jazz zasługuje na osobną sekcję, ponieważ jest zintegrowanym, kompleksowym zestawem narzędzi i aplikacji dla przedsiębiorstw tworzących oprogramowanie. Centrum zarządzania platformą jazz jest serwer stanowiący repozytorium danych i ośrodek dowodzenia. Platforma składa się zarówno udostępnionych na serwerze usług sieciowych oraz aplikacji instalowanych lokalnie, łączących się ze zdalnym serwerem (tzw. rich-client applications). 

  \section{Problemy z istniejącymi rozwiązaniami}

      W poprzedniej sekcji omówiono wybrane narzędzia z wielu dostępnych na rynku aplikacji wspomagających zarządzanie wymaganiami. !! Analiza zastosowanych rozwiązań zdecydowanie prowokuje do przemyśleń i nasuwa na myśl serię pytań związanych motywacjami jakimi kierują się twórcy tego typu oprogramowania. 

      Analizując istniejące rozwiązania, nie można oprzeć się wrażeniu, że wszystkie przystosowane są do dużych, korporacyjnych projektów. Większość aplikacji powiela rozwiązania znane i sprawdzone funkcjonalności. W konsekwencji trudnym zadaniem jest znalezienie wyróżniających się elementów wśród oferowanych możliwości tych produktów. 

      Proces definiowania wymagań w niewielkich (budżet max w granicach 100tyś euro - ?!) projektach jest w dużej mierze procesem twórczym. Często pojawienie się nowego wymagania jest inicjowane z kilku heterogenicznych źródeł. Nierzadko zdarza się również, że to samo wymaganie jest różnie komunikowane przez wiele źródeł. W niewielkich projektach, gdzie często jedna osoba, lub mały zespół odpowiedzialny jest za specyfikację wymagań, wymagania przybierają postać wiadomości email, rozmów telefonicznych, notatek ze spotkań i formalnych dokumentów. Zadaniem osoby lub zespołu odpowiedzialnego za specyfikację wymagań, jest przefiltrowanie cząstkowych informacji oraz ekstrakcja i udokumentowanie wymagań. Żadne narzędzie (przynajmniej na razie) nie zastąpi skutecznie pracy człowieka nad wyłuskaniem wszystkich wymagań. 

      !!!!Wymagania trzymane są w dokumentach procesorów tekstowych - trudno dostępne, ciężko analizować!!!

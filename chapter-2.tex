\chapter{Stan faktyczny w inżynierii oprogramowania}

  W poprzednim rozdziale nakreślono temat przewodni niniejszej pracy. Poruszono problem dostępności i użyteczności istniejących narzędzi wspomagających zarządzanie wymaganiami. Zaprezentowano także autorską koncepcję rozwiązania, mającego na celu usprawnienie procesu tworzenia specyfikacji wymagań użytkownika. 

  Ten rozdział zadedykowany jest analizie stanu faktycznego w zakresie inżynierii oprogramowania w różnych organizacjach.  

  \section{Źródła wiedzy}

  Inżynieria wymagań stanowi fundament wiedzy pozwalającej na realizację wysokiej jakości projektów informatycznych w założonym budżecie i czasie. 

  Istnieje obszerna literatura podejmująca tematy stricte związane z inżynierią wymagań (Gause and Weinberg 1989; Jackson 1995; Kovitz 1999; Robertson and Robertson 1999a; Sommerville and Sawyer 1997), jednak praktyczne doświadczenia oraz liczne dostępne badania wskazują na istotne problemy z wprowadzaniem teorii w życie \cite{Boehm06, Niku00, Standish94}.

  Znany raport opublikowany przez Standish Group International \cite{Standish94} prezentuje dramatyczną rzeczywistość, gdzie tylko niewielka część projektów kończy się częściowym lub całkowitym sukcesem. Wyniki raportu wskazują na najważniejsze powody niepowodzeń projektów jak brak włączenia użytkowników w proces budowy oprogramowania, niekompletne i nieudokumentowane wymagania, wiecznie zmieniający się zakres projektów czy brak wsparcia kadry zarządzającej.      
  
  W badaniu Nikula et al. (2000) przeprowadzonym wśród małych i średnich przedsiębiorstw w Finlandii, autorzy wskazują na alarmujący brak wiedzy dotyczącej wdrażania procesów inżynierii wymagań. Jednocześnie, wnioski z badania wskazują na konieczność zajęcia się problemami inżynierii wymagań na poziomie korporacyjnym, poprzez wdrażanie, optymalizację i automatyzację procesów IW w małych i średnich przedsiębiorstwach \cite{Niku00}.
  
  Z kolei badanie Quispe et al. (2010), również dotyczące małych i średnich przedsiębiorstw, zwraca szczególną uwagę na brak lub nienależytą komunikację z klientem, adresowanie nieodpowiednich problemów klienta, wadliwe specyfikacje oraz stosowanie praktyk inżynierii wymagań ,,ad hoc'' \cite{Quispe10}.
  
  Powyższe badania zajmują się głównie czynnikami niepowodzeń projektów. W badaniu Fern\'{a}ndez et al. (2012) przeanalizowano 12 projektów zakończonych sukcesem. Wnioski z tego badania sugerują, że wiele ze zidentyfikowanych czynników wpływających na niepowodzenia projektów można z powodzeniem rozwiązać zanim staną się realnym zagrożeniem. Ponadto zauważono, że klienci często mają znaczny wpływ na parametry związane z niepowodzeniem projektu. 

  Analiza wyników powyższych badań w kontekście tej pracy sugeruje przede wszystkim konieczność zwrócenia szczególnej uwagi na poziom dojrzałości organizacji w aspekcie inżynierii wymagań. Nie wszystkie projekty kończą się porażką, a doświadczenia pokazują, że im bardziej dojrzałe procesy inżynierii wymagań w organizacji i odpowiednia ,,kultura'' firmy, tym większe są szanse na realizację projektu zwieńczoną sukcesem. 

  Podsumowując, żaden system informatyczny ani narzędzie nie rozwiąże wszystkich problemów związanych z inżynierią oprogramowania. Na organizacji wytwarzającej oprogramowanie ciąży odpowiedzialność usprawniania procesów inżynierii wymagań, edukacji pracowników i zasięgania wiedzy ekspertów. Sukcesy w branży oprpgramowania można osiągnąć tylko poprzez ciągłą naukę, konsekwentną realizację procesów i ich usprawnianie na poziomie korporacyjnym.

\chapter{Istniejące rozwiązania}


  W tym rozdziale zostanie przeprowadzona analiza rynku oprogramowania wspierającego fazę analizy. Druga część rozdziału zajmie się identyfikacją najistotniejszych problemów jakie posiadają istniejące rozwiązania. 

  \section{Analiza rynku systemów wsparcia zbierania wymagań}

    Na rynku nie brakuje systemów wsparcia procesu zbierania wymagań. Dostępne rozwiązania oferowane są w zasadzie w trzech różnych modelach. Najpopularniejszą ostatnio architekturą jest Software As A Service, czyli aplikacja internetowa zainstalowana na serwerach twórcy oprogramowania. Również klasyczne aplikacje okienkowe, które należy zainstalować na komputerze klienta nadal cieszą się dużą popularnością. Należy jednak zaznaczyć, iż w przeważającej większości są to starsze systemy, nierzadko implementowane jeszcze przed popularyzacją zaawansowanych aplikacji webowych. Niektóre firmy oferują także platformy w architekturze klient-serwer, wymagające istnienia zarówno centralnego serwera - repozytorium w sieci jak i desktopowych aplikacji klienckich. W tej sekcji zostaną przedstawione i porównane wybrane aplikacje z każdej z trzech kategorii. 

    \subsection{Model SaaS (Sowtware as a Service)}

      % Największym sukcesem w modelu SaaS zdaje się być platforma Force.com stworzona przez Salesforce (system klasy ERP).

      W ostatnim czasie, wraz z popularyzacją i rozwojem technologii internetowych znacznie wzrosły techniczne możliwości aplikacji dostępnych z poziomu przeglądarki internetowej. Tendencja ta sprzyja powstawaniu licznych aplikacji, adresujących bardzo specyficzne problemy, często w obrębie ściśle określonego segmentu rynku. Jednym z przykładów takiego wąskiego segmentu mogą być np. aplikacje do zarządzania projektami online. Wśród innych popularnych rozwiązań znajdują się takie produkty jak basecamp.com, polski nozbe.com czy unfuddle.com. Innym przykładem mogą być aplikacje wspierające proces rekrutacji (humanway.com, recruiterbox.com, jobvite.com). Na uwagę zasługują także aplikacje wspierające rezerwacje hoteli, niezależnych pokoi i mieszkań jak airbnb.com, booking.com czy b\&b.com. Naturalnie powyższe przykłady dotyczą tylko skrawka możliwych do zagospodarowania segmentów. W rzeczywistości, bardzo ciężko znaleźć wertykalny rynek, który jest niezagospodarowany serwisami, oferującymi różne podejścia do rozwiązania problemów danego segmentu. 

      Powodem takiego stanu rzeczy jest szeroko pojęta popularyzacja internetu oraz przenoszenie się do sieci firm tworzących opgrogramowanie. Dystrybucja oprogramowania w formule SaaS ma wiele zalet w stosunku do klasycznych aplikacji "desktopowych". W szczególności pomięty jest w tym przypadku proces rozprowadzania aplikacji do klientów za pomocą sieci stacjonarnych sklepów z oprogramowaniem. Zaimplementowane rozwiązanie, jest gotowe do użycia z chwilą udostępnienia w internecie. Udostępnienie aplikacji na serwerach twórcy oprogramowania daje nieograniczone możliwości wprowadzania nowych funkcjonalności i poprawek. Z kolei monitorowanie zachowań klientów pozwala niemalże w czasie rzeczywistym odpowiadać na potrzeby użytkowników. 

      W związku z licznymi zaletami modelu SaaS, w internecie powstało wiele aplikacji próbujących zaadresować problemy związane z procesem zbierania i dokumentowania wymagań użytkownika. Do najciekawszych rozwiązań można zaliczyć gatherspace (http://gatherspace.com/), tracecloud.com oraz accompa.com.

      \subsubsection{Gatherspace}
        Gatherspace dzieli wymagania na wymagania biznesowe oraz wymagania funkcjonalne (software requirements). Dostępne są osobne widoki dla obu typów wymagań. Wymagania można katalogować w definiowane przez użytkownika pakiety.

        \begin{figure*}[t]
          \centering
          \includegraphics[width=1.0\textwidth]{img/gatherspace_1.pdf}
          \caption{Gatherspace.com - business requirements}
        \end{figure*}

        \begin{figure*}[t]
          \centering
          \includegraphics[width=1.0\textwidth]{img/gatherspace_2.pdf}
          \caption{Gatherspace.com - software requirements}
          \label{fig:gatherspace_2}
        \end{figure*}
        

        \begin{figure*}[t]
          \centering
          \includegraphics[width=1.0\textwidth]{img/gatherspace_5.pdf}
          \caption{Gatherspace.com - dodawanie wymagania ze szczegółami}
          \label{fig:gatherspace_5}
        \end{figure*}

        \begin{figure*}[t]
          \centering
          \includegraphics[width=1.0\textwidth]{img/gatherspace_6.pdf}
          \caption{Gatherspace.com - tooltip ze szczegółami wymagania}
          \label{fig:gatherspace_6}
        \end{figure*}

        \begin{figure*}[t]
          \centering
          \includegraphics[width=1.0\textwidth]{img/gatherspace_7.pdf}
          \caption{Gatherspace.com - widok szczegółów wymagania}
          \label{fig:gatherspace_7}
        \end{figure*}

        \begin{figure*}[t]
          \centering
          \includegraphics[width=1.0\textwidth]{img/gatherspace_8.pdf}
          \caption{Gatherspace.com - widok szczegółów wymagania}
          \label{fig:gatherspace_8}
        \end{figure*}

        \begin{figure*}[t]
          \centering
          \includegraphics[width=1.0\textwidth]{img/gatherspace_9.pdf}
          \caption{Gatherspace.com - wymaganie przygotowane do druku}
          \label{fig:gatherspace_9}
        \end{figure*}

        Na Rysunku \ref{fig:gatherspace_2} przedstawiono widok wymagań funkcjonalnych. Dodawanie wymagań przebiega w prosty sposób. Po wpisaniu tytułu wymagania i zatwierdzeniu pola formularza, element zostaje dodany do listy bez przeładowania strony.

        Wymaganie może zostać dodane na dwa sposoby - podając jedynie tytuł lub wybierając opcję ,,Add with detail''. Wówczas pojawia się ,,popup'' umożliwiający dodanie zarówno treści jak i opisu wymagania (patrz Rys. \ref{fig:gatherspace_5}).

        Wymaganie posiadające opis, zaopatrzone jest w specjalną ikonkę po lewej stronie. Użytkownik wskazujący myszką na tę ikonkę może podejrzeć treść wymagania, bez konieczności otwierania jego szczegółów. Jest to dość przydatna funkcjonalność, pozwalająca szybko zapoznać się pobieżnie z danym wymaganiem. Rysunek \ref{fig:gatherspace_6} zawiera zrzut ekraniu prezentujący tę funkcjonalność. 
        
        Widok szczegółów wymagania jest dość prosty i przejrzysty (Rys. \ref{fig:gatherspace_7}). Do dyspozycji użytkownika jest formularz zawierający pola z parametrami wymagania, takimi jak m.in.: priorytet, status, poziom złożoności, pakiet do którego należy, przez kogo wymaganie zostało dodane oraz kto lub co jest źródłem wymagania.

        Dodawanie przypadków użycia jest wydzielone do osobnego widoku i funkcjonuje jako osobny zestaw formularzy (Rys. \ref{fig:gatherspace_8}). 

        Istnieje możliwość łączenia przypadków użycia z wymaganiami poprzez wskazanie konkretnego wymagania w osobnym widoku. 

        Zarówno przypadki użycia jak i wymagania mogą zostać wyświetlone w formie przyjaznej do wydruku. Dla przykładu zamieszczono zrzut ekranu obrazujący jeden przypadek użycia w formie do wydruku, na Rys. \ref{fig:gatherspace_9}

         

      \subsubsection{Tracecloud}
      \subsubsection{Accompa}

    \subsection{Aplikacje desktopowe}

      Model SaaS, mimo wszystkich swoich zalet, posiada również wady. Głównym powodem kontrowersji jest konieczność powierzenia danych biznesowych firmie udostępniającej narzędzie na swoich serwerach. Dla dużych korporacji, pilnie strzegących swoich tajemnic handlowych, takie rozwiązanie, może być nie do zaakceptowania ze względów bezpieczeństwa. Mimo potencjalnej możliwośći uniknięcia kosztów budowy i utrzymania infrastruktury sprzętowo-software'owej ryzyko związane z brakiem kontroli nad powierzonymi danymi jest często zbyt wielkie. 

      W klasycznych aplikacjach okienkowych, odpowiedzialność w zakresie zabezpieczenia danych spoczywa na samej korporacji i jej pracowniku. Dzięki temu, nadal istnieje zapotrzebowanie na oprogramowanie instalowane na lokalnym dysku użytkownika. Przykładami takich aplikacji są m.in. IBM DOORS, IBM Rational RequisitePRO oraz seapine.com. 

      \subsubsection{RequisitePRO}
      \subsubsection{DOORS}
      \subsubsection{seapine}

    \subsection{Platforma IBM .jazz}

      Platforma IBM jazz zasługuje na osobną sekcję, ponieważ jest zintegrowanym, kompleksowym zestawem narzędzi i aplikacji dla przedsiębiorstw tworzących oprogramowanie. Centrum zarządzania platformą jazz jest serwer stanowiący repozytorium danych i ośrodek dowodzenia. Platforma składa się zarówno udostępnionych na serwerze usług sieciowych oraz aplikacji instalowanych lokalnie, łączących się ze zdalnym serwerem (tzw. rich-client applications). 

  \section{Problemy z istniejącymi rozwiązaniami}

      W poprzedniej sekcji omówiono wybrane narzędzia z wielu dostępnych na rynku aplikacji wspomagających zarządzanie wymaganiami. !! Analiza zastosowanych rozwiązań zdecydowanie prowokuje do przemyśleń i nasuwa na myśl serię pytań związanych motywacjami jakimi kierują się twórcy tego typu oprogramowania. 

      Analizując istniejące rozwiązania, nie można oprzeć się wrażeniu, że wszystkie przystosowane są do dużych, korporacyjnych projektów. Większość aplikacji powiela rozwiązania znane i sprawdzone funkcjonalności. W konsekwencji trudnym zadaniem jest znalezienie wyróżniających się elementów wśród oferowanych możliwości tych produktów. 

      Proces definiowania wymagań w niewielkich projektach jest w dużej mierze procesem twórczym. Często pojawienie się nowego wymagania jest inicjowane z kilku heterogenicznych źródeł. Nierzadko zdarza się również, że to samo wymaganie jest różnie komunikowane przez wiele źródeł. W niewielkich projektach, gdzie często jedna osoba, lub mały zespół odpowiedzialny jest za specyfikację wymagań, wymagania przybierają postać wiadomości email, rozmów telefonicznych, notatek ze spotkań i formalnych dokumentów. Zadaniem osoby lub zespołu odpowiedzialnego za specyfikację wymagań, jest przefiltrowanie cząstkowych informacji oraz ekstrakcja i udokumentowanie wymagań. Żadne narzędzie (przynajmniej na razie) nie zastąpi skutecznie pracy człowieka nad wyłuskaniem wszystkich wymagań. 

      !!!!Wymagania trzymane są w dokumentach procesorów tekstowych - trudno dostępne, ciężko analizować!!!
